\chapter*{Conclusions}
\lhead[\fancyplain{}{\bfseries\thepage}]{\fancyplain{}{\bfseries\rightmark}}

In this work we proposed a theoretical model for cell differentiation.
Since this biological process is governed by Gene Regulatory Networks, this networks can be modelled by Random Boolean Networks, in which each gene can be represented by  node which can be "on" or "off".
The process of differentiation is a multistable dynamical system, and involves different type of cells, but all of these start from one unique type of cell: the stem cells. The complexity of this process lays in the fact that cells "can" decide if transforming in one type of cell with respect one other, and for this reason seems phesible that if a network for a type of cell is active, it may hinibit the network for a different type of cell.
This process concerns also the birth of cancer cells: cancers cell can be governed by a specific type of regulatory network, which often is inactive, but due to some envinromental noise and stimuli it can be activated and gives an irreversible process of production of cancer cells. This can be seen as a local minima of the Waddington potential, which is impossible to escape.
The role of noise in this model is crucial, but it is well known that biological process are indeed very dependent to external noise.
Today, constructing Gene Regulatory Networks from sequencing data is still impossible, so future studies on Random Boolean Networks may give a way on a deeper understanding of these. 


\chapter{Immunosystem and T-Cells}\label{tcells}
\lhead[\fancyplain{}{\bfseries\thepage}]{\fancyplain{}{\bfseries\rightmark}}

In this Chapter we briefly explain what T-cells are and why they are important for immune system.

\section{T-Cells}

\emph{T-cell}, also called \emph{T lymphocyte}, is a type of leukocyte (white blood cell) that is an essential part of the immune system. T-cells are one of two primary types of lymphocytes (B cells being the second type) that determine the specificity of immune response to antigens (foreign substances) in the body \cite{K39}.
T-cells originate in the bone marrow and mature in the thymus. In the thymus, T-cells multiply and differentiate into different type of cells: \emph{T helper}, \emph{regulatory T-cells} and \emph{cytotoxic T-cells}; further, during their life they can become \emph{memory T-cells}. They are then sent to peripheral tissues or circulate in the blood or lymphatic system. 
In short words, their main roles are:
\begin{itemize}
\centering
\item \textbf{T helper}: once stimulated by the appropriate antigen, helper T-cells secrete chemical messengers called \emph{cytokines}, which stimulate the differentiation of B cells into plasma cells (antibody-producing cells). 

\item \textbf{Regulatory T-cells}: act to control immune reactions, hence their name. 

\item \textbf{Cytotoxic T-cells}: they are activated by various cytokines, bind to and kill infected cells and cancer cells.
\end{itemize} 



Because the body contains millions of T and B cells, many of which carry unique receptors, it can respond to virtually any antigen.

Despite the structural similarities, the receptors on T-cells function differently from those on B cells. The functional difference underlies the different roles played by B and T-cells in the immune system. B cells secrete antibodies to antigens in blood and other body fluids, but T-cells cannot bind to free-floating antigens. Instead they bind to fragments of foreign proteins that are displayed on the surface of body cells. Thus, once a virus succeeds in infecting a cell, it is removed from the reach of circulating antibodies only to become susceptible to the defense system of the T-cell.

Some T-cells recognize class I MHC molecules on the surface of cells; others bind to class II molecules. Cytotoxic T-cells destroy body cells that pose a threat to the individual—namely, cancer cells and cells containing harmful microorganisms. Helper T-cells do not directly kill other cells but instead help activate other white blood cells (lymphocytes and macrophages), primarily by secreting a variety of cytokines that mediate changes in other cells. The function of regulatory T-cells is poorly understood. To carry out their roles, helper T-cells recognize foreign antigens in association with class II MHC molecules on the surfaces of macrophages or B cells. Cytotoxic T-cells and regulatory T-cells generally recognize targeT-cells bearing antigens associated with class I molecules. Because they recognize the same class of MHC molecule, cytotoxic and regulatory T-cells are often grouped together; however, populations of both types of cells associated with class II molecules have been reported. Cytotoxic T-cells can bind to virtually any cell in the body that has been invaded by a pathogen.

T-cells have another receptor, or coreceptor, on their surface that binds to the MHC molecule and provides additional strength to the bond between the T-cell and the targeT-cell. Helper T-cells display a coreceptor called CD4, which binds to class II MHC molecules, and cytotoxic T-cells have on their surfaces the coreceptor CD8, which recognizes class I MHC molecules. These accessory receptors add strength to the bond between the T-cell and the targeT-cell.
The T-cell receptor is associated with a group of molecules called the CD3 complex, or simply CD3, which is also necessary for T-cell activation. These molecules are agents that help transduce, or convert, the extracellular binding of the antigen and receptor into internal cellular signals; thus, they are called signal transducers. Similar signal transducing molecules are associated with B-cell receptors.

\section{Life cycle of T lymphocytes}
When T-cell precursors leave the bone marrow on their way to mature in the thymus, they do not yet express receptors for antigens and thus are indifferent to stimulation by them. Within the thymus the T-cells multiply many times as they pass through a meshwork of thymus cells. In the course of multiplication they acquire antigen receptors and differentiate into helper or cytotoxic T-cells. As mentioned in the previous section, these cell types, similar in appearance, can be distinguished by their function and by the presence of the special surface proteins, CD4 and CD8. Most T-cells that multiply in the thymus also die there. This seems wasteful until it is remembered that the random generation of different antigen receptors yields a large proportion of receptors that recognize self antigens (i.e. molecules present on the body's own constituents) and that mature lymphocytes with such receptors would attack the body’s own tissues.

Most such self-reactive T-cells die before they leave the thymus, so that those T-cells that do emerge are the ones capable of recognizing foreign antigens. These travel via the blood to the lymphoid tissues, where, if suitably stimulated, they can again multiply and take part in immune reactions. The generation of T-cells in the thymus is an ongoing process in young animals. In humans large numbers of T-cells are produced before birth, but production gradually slows down during adulthood and is much diminished in old age, by which time the thymus has become small and partly atrophied. Cell-mediated immunity persists throughout life, however, because some of the T-cells that have emerged from the thymus continue to divide and function for a very long time.

\section{Activation of T lymphocytes}
Helper T-cells do not directly kill infected cells, as cytotoxic T-cells do. Instead they help activate cytotoxic T-cells and macrophages to attack infected cells, or they stimulate B cells to secrete antibodies. Helper T-cells become activated by interacting with antigen-presenting cells, such as macrophages. Antigen-presenting cells ingest a microbe, partially degrade it, and export fragments of the microbe (i.e. antigens) to the cell surface, where they are presented in association with class II MHC molecules. A receptor on the surface of the helper T-cell then binds to the MHC-antigen complex. But this event alone does not activate the helper T-cell. Another signal is required, and it is provided in one of two ways: either through stimulation by a cytokine or through a costimulatory reaction between the signaling protein, B7, found on the surface of the antigen-presenting cell, and the receptor protein, CD28, on the surface of the helper T-cell. If the first signal and one of the second signals are received, the helper T-cell becomes activated to proliferate and to stimulate the appropriate immune cell. If only the first signal is received, the T-cell may be rendered anergic, that is, unable to respond to antigen.

Once the initial steps of activation have occurred, helper T-cells synthesize other proteins, such as signaling proteins and the cell-surface receptors to which the signaling proteins bind. These signaling molecules play a critical role not only in activating the particular helper T-cell but also in determining the ultimate functional role and final differentiation state of that T-cell. For example, the helper T-cell produces and displays IL-2 receptors on its surface and also secretes IL-2 molecules, which bind to these receptors and stimulate the helper T-cell to grow and divide.
The overall result of helper-T-cell activation is an increase in the number of helper T-cells that recognize a specific foreign antigen, and several T-cell cytokines are produced. The cytokines have other consequences, one of which is that IL-2 allows cytotoxic or regulatory T-cells that recognize the same antigen to become activated and to multiply. Cytotoxic T-cells, in turn, can attack and kill other cells that express the foreign antigen in association with class I MHC molecules, which (as explained above) are present on almost all cells. So, for example, cytotoxic T-cells can attack target T-cells that express antigens made by viruses or bacteria growing within them. Regulatory T-cells may be similar to cytotoxic T-cells, but they are detected by their ability to suppress the action of B cells or even of helper T-cells (perhaps by killing them). Regulatory T-cells thus act to damp down the immune response and can sometimes predominate so as to suppress it completely.

\section{Immunotherapy}
Early attempts to harness the immune system to fight cancer involved tumour-associated antigens, proteins that are present on the outer surface of tumour cells. Antigens are recognized as “foreign” by circulating immune cells and thereby trigger an immune response. However, many tumour antigens are altered forms of proteins found naturally on the surface of normal cells; in addition, those antigens are not specific to a certain type of tumour but are seen in a variety of cancers. Despite the lack of tumour specificity, some tumour-associated antigens can serve as targets of attack by components of the immune system. For instance, antibodies can be produced that recognize a specific tumour antigen, and those antibodies can be linked to a variety of compounds (such as chemotherapeutic drugs and radioactive isotopes) that damage cancer cells. In this way the antibody delivers the therapeutic agent directly to the tumour cell. In other cases a chemotherapeutic agent attached to an antibody destroys cancer cells by interacting with receptors on their surfaces that trigger apoptosis.

T-cells themselves may be engineered to recognize, bind to, and kill cancer cells. For example, in an experimental treatment for chronic lymphocytic leukemia, researchers designed a virus to induce the expression on patient T-cells of antibody receptors that identified and attached to antigens on malignant B cells and that activated the T-cells, prompting them to destroy the B cells. T-cells removed from patient blood were incubated with the virus and following infection were infused back into the patient. A portion of the engineered cells persisted as memory T-cells, retaining functionality and suggesting that the cells possessed long-term activity against cancer cells.

A similar T-cell therapy, known as \emph{chimeric antigen receptor T-cells} (CAR-T), in which T-cells isolated from a patient’s blood are genetically engineered to specifically identify and target cancer cells and then are infused back into the patient, has been used in the treatment of certain forms of leukemia, including acute lymphocytic leukemia, as well as B-cell lymphoma. The addition, via genetic engineering, of a unique receptor to the T-cell surface that is capable of recognizing a molecule known as MR1, found on cells from a variety of different cancer types, has opened the possibility of expanding CAR-T to the treatment of solid tumours, in addition to cancers of the blood.



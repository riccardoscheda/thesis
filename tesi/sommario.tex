\chapter*{Sommario}


La cellula vivente è un sistema complesso governato da molti processi che non sono ancora stati compresi: il processo di differenziazione cellulare è uno di questi.
La differenziazione cellulare è il processo in cui le cellule di un tipo specifico si riproducono e danno origine a diversi tipi di cellule.
La differenziazione cellulare è regolata dai cosiddetti Gene Regulatory Networks (GRN).
Un GRN è una raccolta di regolatori molecolari che interagiscono tra loro e con altre sostanze nella cellula per governare i livelli di espressione genica di mRNA e proteine.
Kauffman propose per la prima volta nel 1969 di modellare GRN attraverso le cosiddette Random Boolean Networks (RBN).
I RBN sono reti in cui ogni nodo può avere solo due possibili valori: 0 o 1, dove ogni nodo rappresenta un gene in GRN che può essere "on" oppure "off".
Queste reti possono modellare GRN perché l'attività di un nodo rappresenta il livello di espressione di un gene nell'intera regolazione.

In questo lavoro di tesi ci avvaliamo di un modello matematico per sviluppare e riprodurre una possibile rete di regolazione genica per i linfociti T del sistema immunitario, che presentano una natura bistabile.



\documentclass[12pt,a4paper]{report}
\usepackage[english]{babel}
%\usepackage{newlfont}
%\usepackage[utf8x]{inputenc}
%\usepackage{pgfplots}
%\usepackage{tikz}
\usepackage{pdfpages}
%\usepackage{standalone}
\usepackage{bm}
\usepackage{newlfont}
\usepackage{color}
\textwidth=450pt\oddsidemargin=0pt
\usepackage{systeme}
%\tikzexternalize 
%\textwidth=450pt\oddsidemargin=0pt
\newcommand{\facciatabianca}{\newpage\shipout\null\stepcounter{page}}

\newcommand{\xtodo}[2][]{\tikzexternaldisable\todo[#1]{#2}\tikzexternalenable}
%
%%%%%%%%%%%%%%%%%%%%%%%%%%%%%%%%%%%%%%%%%libreria per impostare il documento
\usepackage{fancyhdr}
%
%%%%%%%%%%%%%%%%%%%%%%%%%%%%%%%%%%%%%%%%%libreria per avere l'indentazione
%%%%%%%%%%%%%%%%%%%%%%%%%%%%%%%%%%%%%%%%%   all'inizio dei capitoli, ...
\usepackage{indentfirst}
%
%%%%%%%%%libreria per mostrare le etichette
%\usepackage{showkeys}
%
%%%%%%%%%%%%%%%%%%%%%%%%%%%%%%%%%%%%%%%%%libreria per inserire grafici
\usepackage{graphicx}
%
%%%%%%%%%%%%%%%%%%%%%%%%%%%%%%%%%%%%%%%%%libreria per utilizzare font
                                        %   particolari ad esempio
                                        %   \textsc{}
\usepackage{newlfont}

%%%%%%%%%%%%%%%%%%%%%%%%%%%%%%%%%%%%%%%%%librerie matematiche
\usepackage{amssymb}
\usepackage{amsmath}
\usepackage{latexsym}
\usepackage{amsthm}
%
%\oddsidemargin=30pt \evensidemargin=20pt%impostano i margini
\hyphenation{}                          %serve per la sillabazione
\theoremstyle{plain}                    %stile corsivo
\newtheorem{teo}{Teorema}[section]      %definizione ambiente teorema
\newtheorem{prop}[teo]{Proposizione}    %definizione ambiente proposizione
\newtheorem{cor}[teo]{Corollario}       %definizione ambiente corollario
\newtheorem{lem}[teo]{Lemma}            %definizione ambiente lemma
\theoremstyle{definition}               %stile roman
\newtheorem{defin}{Definizione}[chapter]%definizione ambiente definizione
\newtheorem{ese}{Esempio}[chapter]      %definizione ambiente esempio
\theoremstyle{remark}                   %stile per osservazioni
\newtheorem{oss}{Osservazione}          %definizione ambiente osservazione
%%%%%%%%%%%%%%%%%%%%%%%%%%%%%%%%%%%%%%%%%comandi per l'impostazione
                                        %   della pagina, vedi il manuale
                                        %   della libreria fancyhdr
                                        %   per ulteriori delucidazioni
\pagestyle{fancy}\addtolength{\headwidth}{20pt}
\renewcommand{\chaptermark}[1]{\markboth{\thechapter.\ #1}{}}
\renewcommand{\sectionmark}[1]{\markright{\thesection \ #1}{}}
\rhead[\fancyplain{}{\bfseries\leftmark}]{\fancyplain{}{\bfseries\thepage}}
\cfoot{}
%%%%%%%%%%%%%%%%%%%%%%%%%%%%%%%%%%%%%%%%%
\linespread{1.3}                        %comando per impostare l'interlinea
%%%%%%%%%%%%%%%%%%%%%%%%%%%%%%%%%%%%%%%%%definisce nuovi comandi
\newcommand{\df}{\displaystyle\frac}    %crea un comando che visualizza le
                                        %   frazioni in modo più esteso
\newcommand{\seq}[1]{\left<#1\right>}   %crea un comando per il "generato"
                                        %   di un insieme, per richiamarlo
                                        %   si può scrivere ad esempio:
                                        %           $\seq{q_1,q_2}$
\begin{document}


\begin{titlepage}
\begin{center}
{{\Large{\textsc{Alma Mater Studiorum $\cdot$ University of  Bologna}}}}
\rule[0.1cm]{15.8cm}{0.1mm}
\rule[0.5cm]{15.8cm}{0.6mm}
\\\vspace{3mm}
{\small{\bf School of Science \\
Department of Physics and Astronomy\\
Master Degree in Physics}}
\end{center}

\vspace{23mm}

\begin{center}
    \LARGE{\bf Modeling cell differentiation using dynamical systems on graphs}\\
\end{center}

\vspace{50mm} \par \noindent

\begin{minipage}[t]{0.47\textwidth}
{\large{\bf Supervisor: \vspace{2mm}\\
Prof. Armando Bazzani}\\\\}
\end{minipage}
%
\hfill
%
\begin{minipage}[t]{0.47\textwidth}\raggedleft
    \textcolor{black}{
        {\large{\bf Submitted by:
            \vspace{2mm}\\
            {Riccardo Scheda}}}
    }
\end{minipage}

\vspace{30mm}

\begin{center}
Academic Year 2019/2020
\end{center}
\end{titlepage}

\facciatabianca
%%%%%%%%%%%%%%%%%%%%%%%%%%%%%%%%%%%%%%%%%non numera l'ultima pagina sinistra
\clearpage{\pagestyle{empty}\cleardoublepage}

\tableofcontents                        %crea l'indice
%%%%%%%%%%%%%%%%%%%%%%%%%%%%%%%%%%%%%%%%%imposta l'intestazione di pagina
\rhead[\fancyplain{}{\bfseries\leftmark}]{\fancyplain{}{\bfseries\thepage}}
\lhead[\fancyplain{}{\bfseries\thepage}]{\fancyplain{}{\bfseries
INDICE}}

%%%%%%%%%%%%%%%%%%%%%%%%%%%%%%%%%%%%%%%%%non numera l'ultima pagina sinistra
\clearpage{\pagestyle{empty}\cleardoublepage}
\listoffigures                          %crea l'elenco delle figure
%%%%%%%%%%%%%%%%%%%%%%%%%%%%%%%%%%%%%%%%%non numera l'ultima pagina sinistra



\chapter*{Introduction}                 %crea l'introduzione (un capitolo
                                        %   non numerato)
%%%%%%%%%%%%%%%%%%%%%%%%%%%%%%%%%%%%%%%%%imposta l'intestazione di pagina
\rhead[\fancyplain{}{\bfseries
INTRODUCTION}]{\fancyplain{}{\bfseries\thepage}}
\lhead[\fancyplain{}{\bfseries\thepage}]{\fancyplain{}{\bfseries
INTRODUCTION}}
%%%%%%%%%%%%%%%%%%%%%%%%%%%%%%%%%%%%%%%%%aggiunge la voce Introduzione
                                        %   nell'indice
\addcontentsline{toc}{chapter}{Introduction}



\chapter{Cell Differentiation }\label{celldiff}
\lhead[\fancyplain{}{\bfseries\thepage}]{\fancyplain{}{\bfseries\rightmark}}


\section{Gene Regulatory Networks}

All steps of gene expression can be modulated, since passage of the transcription of DNA to RNA, to the post-translational modification of the protein
produced. Hence, gene expression is a complex process regulated at several stages in the synthesis of proteins. In addition to the DNA transcription regulation, the expression of a gene may be controlled during RNA processing and transport (in eukaryotes), RNA translation, and the post-translational modification of proteins. This gives rise to genetic regulatory systems structured by networks of regulatory interactions between DNA, RNA, proteins and other molecules [6]: a complex network termed as a gene regulatory
network (GRN). Some, noteworthy, kind of proteins are the transcription factors that bind to specific DNA sequences in order to regulate the
expression of a given gene. The power of transcription factors resides in their ability to activate and/or repress transcription of genes. The activation of
a gene is also referred to positive regulation, while the negative regulation
identifies the inhibition of the gene.
The regulation of gene expression is essential for the cell, because it
allows to control the internal and external functions of the cell. Furthermore,
in multicellular organisms, gene regulation drives the processes of cellular
differentiation and morphogenesis, leading to the creation of different cell
types that possess different gene expression profiles, and these last therefore
produce different proteins that have different ultrastructures that suit them
to their functions (though they all possess the genotype, which follows the
same genome sequence) 4 . Therefore, with few exceptions, all cells in an
organism contain the same genetic material [6], and hence the same genome
(the haploid set of chromosomes of a cell). The difference between the cells
are emergent and due to regulatory mechanisms which can turn on or off
genes. Two cells are different, if they have different subsets of active genes.

\section{Cell Differentiation}
Cell differentiation is the process whereby stem cells become progressively
more specialized. The differentiation process occurs both during the devel-
opment of a multicellular organism and during tissue repair and cell turnover
in the adulthood. Gene expression, and therefore its regulatory mechanisms,
plays a critical role in cell differentiation; as described in the previous section.
Stem cells are undifferentiated biological cells which can both reproduce
themselves, self-renewal ability, and differentiate into specialized cells, po-
tency.

The principles underlying cellular differentiation remain among the most
enigmatic in biology. We are required to explain the spontaneous generation of a multiplicity of cell types from the single zygote, to deduce a natural
tendency of a system to become increasingly heterogeneous, then to stop
differentiating.

Among the important characteristics of cell differentiation are: initiation
of change; stabilization of change after cessation of stimulus; the efficacy of
many substances, exogenous and endogenous, as inductive stimuli; a limit
of five or six as the number of cell types which may differentiate directly from
any cell type; progressive limitation in the number of developmental path-
ways open to any small region of the embryo; restricted periods during which
a cell is competent to respond to an inductive stimulus; the discreteness of
cell types, that is, the mutually exclusive constellations of properties by
which cells differ; a requirement for a minimal and preferably heterogeneous
cell mass to initiate differentiation in many instances, and to maintain it in
some; the occurrence of metaplasia between undifferentiated cell types, or
from an undifferentiated type to a specialized type, but the lack of metaplasia
(the isolation) between specialized cell types; and the cessation of differentia-
tion (Grobstein, 1959).
9).
I believe many aspects of differentiation to be deducible from the typical
behavior of randomly built genetic nets.
Cells are thought to differ due to differential expression of, rather than
structural loss of, the genes. Differential activity of the genes raises at least
two questions which are not always carefully distinguished: the capacity of
the genome to behave in more than one mode; and mechanisms which insure
the appropriate assignment of these modes to the proper cells. The second
presumes the first.
Randomly assembled nets of binary elements behave in a multiplicity
of
distinct modes. Different state cycles embodied in a net are isolated from
each other, for no state may be on two cycles. Thus, a multiplicity
of state
cycles, each a different temporal sequence of genetic activity, is to be expected
in randomly constructed genetic nets. It seems reasonable to identify one cell
type with one state cycle. To the extent that this binary model, in which the
expression of the “gene” is potentially reversible at each clocked moment, is
accurate, it demonstrates the common occurrence of multiple modes of
behavior in a genetic system.
Tf this identification is reason

\chapter{Cancer attractors}
\lhead[\fancyplain{}{\bfseries\thepage}]{\fancyplain{}{\bfseries\rightmark}}

%\newcommand{\folder}{/path/to/folder}
\newenvironment{sistema}%
{\left\lbrace\begin{array}{@{}l@{}}}%
{\end{array}\right.}
In questo capitolo viene fatta una piccola introduzione al modello classico preda-predatore di Lotka-Volterra.


\section{}


\chapter{Random Boolean Networks}
\lhead[\fancyplain{}{\bfseries\thepage}]{\fancyplain{}{\bfseries\rightmark}}


In this chapter we explain the basic concepts of Random Boolean Network proposed for the first time by Kauffman.

\section{Random Boolean Networks}
Random Boolean networks (RBNs) were introduced in
1969 by S. Kauffman as a simple model of genetic systems.
Each gene was represented by a node
that has two possible states, “on” (corresponding to a
gene that is being transcribed) and “off” (corresponding
to a gene that is not being transcribed). There are altogether N nodes, and each node receives input from K
randomly chosen nodes, which represent the genes that
control the considered gene. Furthermore, each node is
assigned an update function that prescribes the state of
the node in the next time step, given the state of its input nodes. This update function is chosen from the set
of all possible update functions according to some probability distribution. Starting from some initial configuration, the states of all nodes of the network are updated
in parallel. Since configuration space is finite and since
dynamics is deterministic, the system must eventually return to a configuration that it has had before, and from
then on it repeats the same sequence of configurations
periodically: it is on an attractor.

\section{The model}
Let's consider a network of $N$ nodes. The state of each node at a time $t$ is given by $\sigma_i(t) \in {0,1}$ with $ i = 1,...,N$.
The $N$ nodes of the network can therefore together assume $2^N$ different states.
The number of incoming links to each node $i$  is denoted by $k_i$ and is drawn
randomly independently from the distribution $P(k_i)$.
The dynamical state of each $\sigma_i(t)$ is updated synchronously by a Boolean function $\Lambda_i$:
$$
\Lambda_i:\{0,1\}^{k_i} \to \{0,1\}
$$ 
An update function specifies
the state of a node in the next time step, given the state
of its $K$ inputs at the present time step. Since each of the
$K$ inputs of a node can be on or off, there are $M = 2^K$ possible input states.
The update function has to specify the new state of a node for each of these input states.
Consequently, there are $2^M$ different update functions.
For example let's consider a network with $K=1$, so all the functions $\Lambda_i$ receives the input from one single node. 
In general each element 
receives inputs from exactly $K$ nodes, so we have a dynamical system defined from:

\begin{equation}
\sigma_i(t+1)=\Lambda_i(\sigma_{i_1}(t),\sigma_{i_2}(t), ...,\sigma_{i_K}(t)).
\end{equation}  

So, the randomness of these network appears at two levels: in the connectivity of the network (which node is linked
to which) and the dynamics (which function is attributed to which node).

\section{Topology}
For a given number $N$ of nodes and a given number
$K$ of inputs per node, a RBN is constructed by choosing
the $K$ inputs of each node at random among all nodes.
If we construct a sufficiently large number of networks in
this way, we generate an ensemble of networks. In this
ensemble, all possible topologies occur, but their statis-
tical weights are usually different. Let us consider the
simplest possible example, $N = 2$ and $K = 1$, shown
in Figure~\ref{fig:rb}. There are 3 possible topologies.



\begin{figure}[h]
\centering
\includegraphics[scale=1]{figurenetworks.pdf}
\caption{Set of all possible networks with $N=2$ and $K=1$.}
\label{fig:rb}
\end{figure}

Topologies (a) and (b) have each the statistical weight $1/4$ in
the ensemble, since each of the links is connected in the
given way with probability $1/2$. Topology (c) has the
weight $1/2$, since there are two possibilities for realizing
this topology: either of the two nodes can be the one
with the self-link.


While the number of inputs of each node is fixed by
the parameter $K$, the number of outputs (i.e. of outgo-
ing links) varies between the nodes. The mean number of
outputs must be $K$, since there must be in total the same
number of outputs as inputs. A given node becomes the
input of each of the N nodes with probability $\frac{K}{N}$ . In
the thermodynamic limit $N \to \infty$ the probability distribution of the number of outputs is therefore a Poisson
distribution:

$$
P_{out}(k) = \frac{K^k}{k!}e^{-K}
$$

\section{Dynamics}

All nodes are updated at the same time
according to the state of their inputs and to their update
function. Starting from some initial state, the network
performs a trajectory in state space and eventually arrives on an \emph{attractor}, where the same sequence of states
is periodically repeated. Since the update rule is deterministic, the same state must always be followed by the
same next state. If we represent the network states by
points in the $2^N$-dimensional state space, each of these
points has exactly one “output”, which is the successor
state. We thus obtain a graph in state space.
The size or length of an attractor is the number of
different states on the attractor. The basin of attraction
of an attractor is the set of all states that eventually
end up on this attractor, including the attractor states
themselves. The size of the basin of attraction is the
number of states belonging to it. The graph of states
in state space consists of unconnected components, each
of them being a basin of attraction and containing an
attractor, which is a loop in state space. The transient
states are those that do not lie on an attractor. They are
on trees leading to the attractors.


Let us illustrate these concepts by studying the small
$K = 1$ network shown in Figure~\ref{fig:rb2}, which consists of 4
nodes:

\begin{figure}[h]
\centering
\includegraphics[scale=1]{figurenetworks2.pdf}
\caption{A small network with $N=4$ and $K=1$.}
\label{fig:rb2}
\end{figure}

If we assign to the nodes 1,2,3,4 the functions invert,
invert, copy, copy, an initial state $1111$ evolves in the
following way:

$$
1111 \to 0011 \to 0100 \to 1111
$$

This is an attractor of period 3. If we interpret the bit se-
quence characterizing the state of the network as a number in binary notation, the sequence of states can also be
written as

$$
15 \to 3 \to 4 \to 15
$$

The entire state space is shown in Figure~\ref{fig:rb3}:
\begin{figure}[h]
\centering
\includegraphics[scale=1]{figurenetworks2.pdf}
\caption{The state space of the network shown in Figure~\ref{fig:rb2}, if
the functions copy, copy, invert, invert are assigned to the four
nodes. The numbers in the squares represent states, and ar-
rows indicate the successor of each state. States on attractors
are shaded.}
\label{fig:rb3}
\end{figure}

There are 4 attractors, two of which are fixed points
(i.e., attractors of length 1). The sizes of the basins of
attraction of the 4 attractors are 6,6,2,2. If the function
of node 1 is a constant function, fixing the value of the
node at 1, the state of this node fixes the rest of the
network, and there is only one attractor, which is a fixed
point. Its basin of attraction is of size 16. If the functions
of the other nodes remain unchanged, the state space
then looks as shown in Figure~\ref{fig:rb4}

\begin{figure}[h]
\centering
\includegraphics[scale=1]{figurenetworks2.pdf}
\caption{The state space of the network shown in Figure~\ref{fig:rb2},
if the functions 1, copy, invert, invert are assigned to the four
nodes.}
\label{fig:rb4}
\end{figure}

Before we continue, we have to make the definition of
attractor more precise: as the name says, an attractor
“attracts” states to itself. A periodic sequence of states
(which we also call cycle) is an attractor if there are states
outside the attractor that lead to it. However, some netaworks contain cycles that cannot be reached from any
state that is not part of it. For instance, if we removed
node 4 from the network shown in Figure 2.2, the state
space would only contain the cycles shown in Figure 2 C,
and not the 8 states leading to the cycles. In the follow-
ing, we will use the word “cycle” whenever we cannot be
confident that the cycle is an attractor.


\section{Applications}

Let us now make use of the definitions and concepts
introduced in this section in order to derive some results
concerning cycles in state space. First, we prove that in
an ensemble of networks with update rule 1 (biased functions) or rule 2 (weighted classes), there is on an average
exactly one fixed point per network. A fixed point is a
cycle of length 1. The proof is slightly different for rule
1 and rule 2. Let us first choose rule 2. We make use of
the property that for every update function the inverted
function has the same probability. The inverted function
has all 1s in the output replaced with 0s, and vice versa.
Let us choose a network state, and let us determine for
which fraction of networks in the ensemble this state is a
fixed point. We choose a network at random, prepare it
in the chosen state, and perform one update step. The
probability that node 1 remains in the same state after
the update, is 1/2, because a network with the inverted
function at node 1 occurs equally often. The same holds
for all other nodes, so that the chosen state is a fixed
point of a given network with probability $2^{−N}$ . This
means that each of the 2 N states is a fixed point in the
proportion $2^{−N}$ of all networks, and therefore the mean
number of fixed points per network is 1. We will see
later that fixed points may be highly clustered: a small
proportion of all networks may have many fixed points,
while the majority of networks have no fixed point.


Next, we consider rule 1. We make now use of the
property that for every update function a function with
any permutation of the input states has the same probability. This means that networks in which state A leads
to state B after one update, and networks in which another state C leads to state B after one update, occur
equally often in the ensemble. Let us choose a network
state with n 1s and $N − n$ 0s. The average number of
states in a network leading to this state after one update
is $2^N p^n (1 − p^{N −n}$ . Now, every state leads equally often
to this state, and therefore this state is a fixed point in
the proportion $p^n (1 − p)^{N−n}$ of all networks. Summation
over all states gives the mean number of fixed points per
network, which is 1.

Finally, we derive a general expression for the mean
number of cycles of length L in networks with $K = 2$
inputs per node. The generalization to other values of $K$
is straightforward. Let $\langle C_L\rangle_N$ denote the mean number
of cycles in state space of length $L$, averaged over the
ensemble of networks of size $N$ . On a cycle of length
L, the state of each node goes through a sequence of
1s and 0s of period L. Let us number the $2^L$ possible
sequences of period L of the state of a node by the index
j, ranging from 0 to $m = 2^L − 1$. Let $n_j$ denote the
number of nodes that have the sequence $j$ on a cycle of
length$ L$, and $(P_L )^j_{l,k}$ the probability that a node that
has the input sequences $l$ and $k$ generates the output
sequence $j$. This probability depends on the probability distribution of update functions.



Then

\begin{equation}
\langle C_L\rangle_N = \frac{1}{L} \sum_{n_j} \frac{N!}{n_0! \dots n_m!} \prod_j \big(\sum_{l,k}\frac{n_ln_k}{N^2}(P_L)^j_{l,k}\big)^{n_j}
\end{equation}

The factor 1/L occurs because any of the L states on the
cycle could be the starting point. The sum is P
over all
possibilities to choose the values {n j } such that j n j =
N . The factor after the sum is the number of different
ways in which the nodes can be divided into groups of the
sizes n 0 , n 1 , n 2 , . . . , n m . The product is the probability
that each node with a sequence j is connected to nodes
with the sequences l and k and has an update function
that yields the output sequence j for the input sequences
l and k. This formula was first given in the beautiful
paper by Samuelsson and Troein [10].


\chapter{Cancer attractors}
\lhead[\fancyplain{}{\bfseries\thepage}]{\fancyplain{}{\bfseries\rightmark}}

%\newcommand{\folder}{/path/to/folder}
\newenvironment{sistema}%
{\left\lbrace\begin{array}{@{}l@{}}}%
{\end{array}\right.}
In questo capitolo viene fatta una piccola introduzione al modello classico preda-predatore di Lotka-Volterra.
\section{The model}
We consider a physical system that can be described by an weighted interaction network among nodes that can assume different
dynamical states (in the case of a gene network the states $\sigma\in [0,1]$ and we have models similar to spin models).
In the simplest case, we introduce a stochastic dynamics using the probability $p_i(t)$ that the node $i$ is in the state $\sigma_i=1$
(then $1-p_i(t)$ is the probability to get $\sigma_i=0$) and we define a linear equation for the probability evolution
\begin{equation}
\dot p_i(t)=\sum_j \mathcal{P}_{ij}p_j(t)-\gamma_i p_i(t)
\label{average}
\end{equation}
where $\mathcal{P}_{ij}$ are transition probability rates and $\gamma_i^{-1}$ defines the mean lifetime of the excited state.
The meaning of the rates $\mathcal{P}_{ij}$ is the rate at which the excited state of the node $j$ increases (or decreases if
$\mathcal{P}_{ij}<0$) the probability of a transition to the excited state of the node $i$. Since $0\le p_i\le 1$ for all $i$, this space should be invariant for the dynamics. This condition depends on the spectral properties of the matrix
\begin{equation}
\mathcal{P}_{ij}-\gamma_j\delta_{ij}
\label{matrix}
\end{equation}
associated to the system. Let consider the case $\mathcal{L}_{ij}\ge 0$ (i.e. we have no inhibitory link),
the first quadrant is clearly invariant and if we define
$$
\sum_i  \mathcal{P}_{ij}=\hat \gamma_j>0
$$
the matrix 
$$
\mathcal{L}_{ij}=\mathcal{P}_{ij}-\hat \gamma_j \delta_{ij}
$$
is a Laplacian matrix and the system (\ref{average}) can be written in the form
$$
\dot p_i(t)=\sum_j \mathcal{L}_{ij}p_j(t)-\Delta \gamma_i p_i(t)\qquad \Delta \gamma_i=\gamma_i-\hat \gamma_i
$$
and by assumption we have $\gamma_i>\hat \gamma_i$. The eigenvalues of the matrix $\mathcal{L}_{ij}$
have all negative real part except the null eigenvalue. It follows that all the eigenvalue of the matrix (\ref{matrix})
has negative real part and the dynamics is a contraction towards the origin:
a stable solution (i.e. without any external stimulus the system relaxes to the $\sigma_i=0$ state).
A non trivial stationary can be achieved only if an external stimulus is inserted
\begin{equation}
\dot p_i(t)=\sum_j \mathcal{P}_{ij}p_j(t)-\gamma_i p_i(t)+\epsilon f_i(t)
\label{average_ext}
\end{equation}
The stationary solution has to satisfy $p_i\in [0,1]$ so that $f_i(t)\ge 0$ otherwise we can have negative probability 
when $p_i\simeq 0$. The case of a Laplacian matrix
$$
\hat \gamma_i=\gamma_i
$$
we get another possible stationary solution for $\mathcal{L}_{ij}p^\ast_j=0$ in the first quadrant and
the subspace $\sum p_i=0$ is invariant and the dynamics is a contraction in this subspace (in general).
Then the system a stable stationary solution even in absence of an external stimulus.\par\noindent
The presence of inhibitory links complicates the model and one has to prove that
\begin{itemize}
\item 1) there exists a physical space: an invariant cone in the first quadrant where the dynamics is a contraction towards
the origin;\par
\item 2) the external stimulus maintains the solution in the physical space.
\end{itemize}
Another solution could be to introduce boundary conditions so that $p_i\ge 0$ in any case (the system is non linear in such a case).
\par\noindent
The eigenvalues of the matrix (\ref{matrix}) define the different relaxation time scale the process and determine its rectivity
to the change of the external stimulus: in a typical problem one consider a slowly varying external stimulus so that the
system could be considered i a quasi stationary state
$$
\sum_j \mathcal{L}_{ij}p_j-\Delta \gamma_i p_i=-\epsilon f_i(t) \qquad \frac{df_i}{dt}\ll 1
$$
the derivative is small with respect to the eigenvalues of th matrix (adiabatic approximation). On the other hand we have the effect
of a correlated noise (we need to introduce a correlation in order have a continuous function $f_i(t)$). The problem is to
study the relation between the solution and the spectral properties of the matrix $\mathcal{L}_{ij}$: we simplify the
equation by assuming $\Delta \gamma_i=\Delta \gamma$ so that if $\lambda$ is an eigenvalue of $\mathcal{L}_{ij}$
then $\lambda-\Delta \gamma$ is an eigenvalue of the matrix (\ref{matrix}) and we assume that the dynamics
is perturbed by
\begin{equation}
\dot p_i(t)=\sum_j\left ( \mathcal{L}_{ij}+\Delta \mathcal{L}_{ij}\right ) p_j(t)-\Delta \gamma p_i(t)+\epsilon f_i(t)
\label{average_p}
\end{equation}
where the perturbation $\Delta\mathcal{L}_{ij}$ is a Laplacian matrix ($\sum_i \Delta\mathcal{L}_{ij}=0$ and we
assume $<\mathcal{L}>=0$) that can
represent an error in the measure of the transition rates $\mathcal{L}_{ij}$ or possible evolution of network due to
in time. In the first case we have an ensemble of transition matrices and we have to study the eigenvalue distribution
due to perturbation and the possible presence of bifurcation phenomena. In the second case we have a stochastic 
differential equation (since $\Delta \mathcal{L}_{ij}(t)$ can be represented as a realization of a stochastic process).
The possible approach are Perturbation Theory, Random Matrix Theory and Statistical Physics Methods for random matrices.
The external signal form the environment  (the environmental node) can be considered in the adiabatic approximation
(to be justified form a biological point of view).\par\noindent
The underlying stochastic process on the graph is defined by assigning the state $\xi_i(t)\in [0,1]$ at each node $i$ according 
to a probability distribution $\pi_i(t)$ that evolves as
$$
\dot \pi_i(t)=\sum_j \mathcal{L}_{ij}\xi_j(t)-\Delta \gamma \xi_i(t)+\epsilon f_i(t)
$$
By discretizing the dynamics for a time step $\Delta t$ we have the evolution
$$
\pi_i(t+\Delta t)=\pi(t)+\sum_j \mathcal{L}_{ij}\xi_j(t)\Delta t-\Delta \gamma \Delta t\xi_i(t)+\epsilon f_i(t)\Delta t
$$
and $\xi(t+\Delta t)$ realized according to the distribution $\pi_i(t+\Delta t)$ (stochastic cellular automata).
The average dynamics is computed by
\begin{eqnarray}
\dot <\pi_i(t)>&=&\sum_j \mathcal{L}_{ij}<\xi_j(t)>-\Delta \gamma <\xi_i(t)>+\epsilon f_i(t)\nonumber \\
&=&\sum_j \mathcal{L}_{ij}p_j(t)-\Delta \gamma p_i(t)+\epsilon f_i(t)=\dot p_i(t)\nonumber
\end{eqnarray}
and we recover the average equation (\ref{average}). But the stochastic dynamics gives information on the applicability
of the average approximation and the variability at the critical states (at bifurcation of the spectrum of $\mathcal{L}$).
The stochastic dynamics can be studied for stochastic connection matrices $\mathcal{L}+\Delta \mathcal{L}$.



\pagestyle{plain}


\chapter*{Appendix A}

\section*{Approssimazione di campo medio}


\begin{thebibliography}{90}             %crea l'ambiente bibliografia
\rhead[\fancyplain{}{\bfseries \leftmark}]{\fancyplain{}{\bfseries
\thepage}}
%%%%%%%%%%%%%%%%%%%%%%%%%%%%%%%%%%%%%%%%%aggiunge la voce Bibliografia
                                        %   nell'indice
\addcontentsline{toc}{chapter}{Bibliography}
%%%%%%%%%%%%%%%%%%%%%%%%%%%%%%%%%%%%%%%%%provare anche questo comando:
%%%%%%%%%%%\addcontentsline{toc}{chapter}{\numberline{}{Bibliografia}}

\bibitem{K39} Janeway, \emph{Immunobiology}, 9th Edition
\bibitem{K40} E. Davidson, M. Levine, \emph{Gene Regulatory Networks}, doi:10.1073/pnas.0502024102, (2005)
\bibitem{K41} A. Wuensche, \emph{Genomic regulation modeled as a network with basins of attraction}, Pacific Symposium on Biocomputing. Pacific Symposium on Biocomputing, (1998)

\bibitem{K49} S. A. Kauffman, \emph{Investigations}, Oxford University
Press, (2000)
\bibitem{K1} S. A. Kauffman, \emph{Metabolic Stability and Epigenesis in
Randomly Constructed Genetic Nets}, J. Theoret. Biol. (1969)
\bibitem{K42} MacArthur S. et al., \emph{Developmental roles of 21 Drosophila transcription factors are determined by quantitative differences in binding to an overlapping set of thousands of genomic regions},doi:10.1186/gb-2009-10-7-r80 (2009)
\bibitem{K7} S. A. Kauffman: J. Theor. Biol., 44, Physica D, 10, 145 (1984)
\bibitem{K43} Peccoud, J. and Ycart, B., \emph{Markovian Modelling of Gene Products Synthesis.}, https://doi.org/10.1006/tpbi.1995.1027, (1995)
\bibitem{K44} T.B. Kepler and T.C. Elston, \emph{Stochasticity in transcriptional regulation: origins, consequences, and mathematical representations}, 10.1016/S0006-3495(01)75949-8 , (2001)
\bibitem{K45} J.M. Pedraza, J. Paulsson, \emph{Effects of molecular memory and bursting on fluctuations in gene expression}, 10.1126/science.1144331, (2008).
\bibitem{K46} Y. Sasai \emph{Cytosystems dynamics in self-organization of tissue architecture}, https://doi.org/10.1038/nature11859, (2013)

\bibitem{K47} B. Zhang and P. G. Wolynes, \emph{Stem cell differentiation as a many-body problem}, https://doi.org/10.1073/pnas.1408561111, (2014)
\bibitem{K48} Zhou et al., \emph{Fast Pyrolysis of Glucose-Based Carbohydrates withAdded NaCl Part 2: Validation and Evaluation of theMechanistic Model},DOI 10.1002/aic.15107, (2016)
\bibitem{K8} B. Derrida, \emph{Random Networks of Automata: A Simple Annealed
Approximat ion.}, (1985)
\bibitem{K5} Drossel B.,\emph{Random Boolean Networks},arXiv:0706.3351 ,(2008)

\bibitem{K3} R.Serra, M. Villani, A. Barbieri, S.A. Kauffman, A. Colacci,\emph{On the dynamics of random Boolean networks subject to noise:
Attractors, ergodic sets and cell types.},J Theor Biol 265: 185–193, (2010)
\bibitem{K2} M. Villani, A. Barbieri, R. Serra,\emph{A Dynamical Model of Genetic Networks for Cell Differentiation}, doi:10.1371/journal.pone.0017703.g001,(2011)

\bibitem{K4} S. Huang, I. Ernberg, S. Kauffman,\emph{Cancer attractors: A systems view of tumors from a gene network
dynamics and developmental perspective}, doi:10.1016/j.semcdb.2009.07.003, (2009)
\bibitem{K6} S. Kauffman, \emph{A proposal for using the ensemble approach to understand
genetic regulatory networks},Journal of Theoretical Biology 230 (2004) 581–590 ,(2004)
\bibitem{K9} M. Ali Al-Radhawi , Nithin S. Kumar, Eduardo D. Sontag , Domitilla Del Vecchio, \emph{Stochastic multistationarity in a model of the hematopoietic
stem cell differentiation network},doi:10.1109/cdc.2018.8619300, (2018)
\bibitem{K10} Cameron P. Gallivan, Honglei Ren and Elizabeth L. Read,\emph{Analysis of Single-Cell Gene Pair
Coexpression Landscapes by
Stochastic Kinetic Modeling Reveals
Gene-Pair Interactions in
Development},doi: 10.3389/fgene.2019.01387 ,(2019)

\bibitem{K11} Jifan Shi, Tiejun Li , Luonan Chen, Kazuyuki Aihara,\emph{Quantifying pluripotency landscape of cell
differentiation from scRNA-seq data by
continuous birth-death process},https://doi.org/10.1371/journal.
pcbi.1007488 ,(2019)
\bibitem{K12} Jin Wang, Kun Zhang, Li Xu, and Erkang Wang ,\emph{Quantifying the Waddington landscape and biological
paths for development and differentiation}, https://doi.org/10.1073/pnas.1017017108 ,(2011)
\bibitem{K13} Waddington CH, \emph{The strategy of the genes: a discussion of some aspects of
theoretical biology}. London: Allen and Unwin, (1957)


\bibitem{K14} B. Drossel,\emph{Random Boolean Networks}, arXiv:0706.3351. (2008)
\bibitem{K15} M. Rybarsch and S. Bornholdt,\emph{On the dangers of Boolean networks:
Activity dependent criticality and threshold networks not faithful to biology}, arXiv:1012.3287v1. (2010)
\bibitem{K16} J. Park and M. E. J. Newman,\emph{The statistical mechanics of networks}, DOI: 10.1103/PhysRevE.70.066117 (2004)
\bibitem{K17} C. Gershenso,\emph{Introduction to Random Boolean Networks}, arXiv:nlin/0408006,(2004)
\bibitem{K17} B. Derrida and H.Flyvbjerg, \emph{The random map model: a disordere model with deterministic dynamics}, J.Physique, (1987)
\bibitem{K19} R. V. Solè, B. Loque, \emph{Phase transitions and antichaos in generalized Kauffman networks},Physics Letters,(1994)
\bibitem{K20} J. T. Lizier, S. Pritam, M. Prokopenko, \emph{Information dynamics in small-world Boolean networks},, (2011)
\bibitem{K21} B. Derrida, \emph{Spin glasses, random boolean networks and simple models of evolution}
\bibitem{K22} A. Rèka and A-L. Barabàsi, \emph{Statistical mechanics of complex networks}, Reviewes of modern physics, Volume 74,(2002)
\bibitem{K23} N. Masuda , M. A. Porter, R. Lambiotte ,\emph{Random walks and diffusion on networks},Physics Reports 716–717 1–58,(2017)
\bibitem{K24} T. Biyikoglu, J. Leydold, P. F. Stadler,\emph{Laplacian Eigenvectors of Graphs}, Springer
\bibitem{K25} Fan R. K. Chung,\emph{Spectral Graph Theory}, CBMS
\bibitem{K26} Sui Huang , Ingemar Ernberg, and Stuart Kauffman,\emph{Cancer attractors: A systems view of tumors from a gene network
dynamics and developmental perspective}, DOI:10.1016/j.semcdb.2009.07.003, (2009)
\bibitem{K27} M. Ali Al-Radhawi, Nithin S. Kumar, Eduardo D. Sontag, Domitilla Del Vecchio ,\emph{Stochastic multistationarity in a model of the hematopoietic
stem cell differentiation network},DOI:10.1109/cdc.2018.8619300.,(2018)
\bibitem{K28} Rushina Shah, Domitilla Del Vecchio,\emph{Reprogramming cooperative monotone dynamical systems},DOI:10.1109/cdc.2018.8618649, (2018)
\bibitem{K29} Atefeh Taherian Fard and Mark A. Ragan, \emph{Modeling the Attractor Landscape of
Disease Progression: a
Network-Based Approach},DOI: 10.3389/fgene.2017.00048, (2017)
\bibitem{K30} Sui Huang, Yan-Ping Guo, Gillian May, Tariq Enver,\emph{Bifurcation dynamics in lineage-commitment in bipotent progenitor cells},Developmental Biology 305, (2007)
\bibitem{K31} Xin-She Yang and Young Z. L. Yang,\emph{Cellular Automata Networks},arXiv:1003.4958 , (2010)
\bibitem{K32} Christopher H. Joyner and Uzy Smilansky, \emph{Dyson’s Brownian-motion model for random matrix
theory - revisited},arXiv:1503.06417,(2015)
\bibitem{K33} Cameron P. Gallivan, Honglei Ren and Elizabeth L. Read,\emph{Analysis of Single-Cell Gene Pair
Coexpression Landscapes by
Stochastic Kinetic Modeling Reveals
Gene-Pair Interactions in
Development} ,doi: 10.3389/fgene.2019.01387,(2019)

\bibitem{K34} Xin Kang, Chunhe Li,\emph{Landscape inferred from gene expression data governs pluripotency in
embryonic stem cells},omputational and Structural Biotechnology Journal 18 (2020) 366–374,(2020)
\bibitem{K35} Genaro J. Martı́nez , Andrew Adamatzky  ,
Bo Chen , Fangyue Chen , Juan C.S.T. Mora ,\emph{Simple networks on complex cellular automata:
From de Bruijn diagrams to jump-graphs},(2017)

\bibitem{K37} Chen L et al., \emph{Biomolecular networks: methods and applications in systems biology}, Wiley, Hoboken ,(2009)

\bibitem{K38} Schnakenberg J., \emph{Network theory of microscopic and macroscopic behavior of master equation systems.}, Reviews of Modern Physics,Vol.48, (1976)
\bibitem{K50} C.W. Gardiner, \emph{Handbook of Stochastic Methods}, Springer, (1985)



\end{thebibliography}
%%%%%%%%%%%%%%%%%%%%%%%%%%%%%%%%%%%%%%%%%non numera l'ultima pagina sinistra
\clearpage{\pagestyle{empty}\cleardoublepage}


\end{document}


\chapter{Cell Differentiation }\label{celldiff}
\lhead[\fancyplain{}{\bfseries\thepage}]{\fancyplain{}{\bfseries\rightmark}}


\section{Gene Regulatory Networks}

All steps of gene expression can be modulated, since passage of the transcription of DNA to RNA, to the post-translational modification of the protein
produced. Hence, gene expression is a complex process regulated at several stages in the synthesis of proteins. In addition to the DNA transcription regulation, the expression of a gene may be controlled during RNA processing and transport (in eukaryotes), RNA translation, and the post-translational modification of proteins. This gives rise to genetic regulatory systems structured by networks of regulatory interactions between DNA, RNA, proteins and other molecules [6]: a complex network termed as a gene regulatory
network (GRN). Some, noteworthy, kind of proteins are the transcription factors that bind to specific DNA sequences in order to regulate the
expression of a given gene. The power of transcription factors resides in their ability to activate and/or repress transcription of genes. The activation of
a gene is also referred to positive regulation, while the negative regulation
identifies the inhibition of the gene.
The regulation of gene expression is essential for the cell, because it
allows to control the internal and external functions of the cell. Furthermore,
in multicellular organisms, gene regulation drives the processes of cellular
differentiation and morphogenesis, leading to the creation of different cell
types that possess different gene expression profiles, and these last therefore
produce different proteins that have different ultrastructures that suit them
to their functions (though they all possess the genotype, which follows the
same genome sequence) 4 . Therefore, with few exceptions, all cells in an
organism contain the same genetic material [6], and hence the same genome
(the haploid set of chromosomes of a cell). The difference between the cells
are emergent and due to regulatory mechanisms which can turn on or off
genes. Two cells are different, if they have different subsets of active genes.

\section{Cell Differentiation}
Cell differentiation is the process whereby stem cells become progressively
more specialized. The differentiation process occurs both during the devel-
opment of a multicellular organism and during tissue repair and cell turnover
in the adulthood. Gene expression, and therefore its regulatory mechanisms,
plays a critical role in cell differentiation; as described in the previous section.
Stem cells are undifferentiated biological cells which can both reproduce
themselves, self-renewal ability, and differentiate into specialized cells, po-
tency.

The principles underlying cellular differentiation remain among the most
enigmatic in biology. We are required to explain the spontaneous generation of a multiplicity of cell types from the single zygote, to deduce a natural
tendency of a system to become increasingly heterogeneous, then to stop
differentiating.

Among the important characteristics of cell differentiation are: initiation
of change; stabilization of change after cessation of stimulus; the efficacy of
many substances, exogenous and endogenous, as inductive stimuli; a limit
of five or six as the number of cell types which may differentiate directly from
any cell type; progressive limitation in the number of developmental path-
ways open to any small region of the embryo; restricted periods during which
a cell is competent to respond to an inductive stimulus; the discreteness of
cell types, that is, the mutually exclusive constellations of properties by
which cells differ; a requirement for a minimal and preferably heterogeneous
cell mass to initiate differentiation in many instances, and to maintain it in
some; the occurrence of metaplasia between undifferentiated cell types, or
from an undifferentiated type to a specialized type, but the lack of metaplasia
(the isolation) between specialized cell types; and the cessation of differentia-
tion (Grobstein, 1959).
9).
I believe many aspects of differentiation to be deducible from the typical
behavior of randomly built genetic nets.
Cells are thought to differ due to differential expression of, rather than
structural loss of, the genes. Differential activity of the genes raises at least
two questions which are not always carefully distinguished: the capacity of
the genome to behave in more than one mode; and mechanisms which insure
the appropriate assignment of these modes to the proper cells. The second
presumes the first.
Randomly assembled nets of binary elements behave in a multiplicity
of
distinct modes. Different state cycles embodied in a net are isolated from
each other, for no state may be on two cycles. Thus, a multiplicity
of state
cycles, each a different temporal sequence of genetic activity, is to be expected
in randomly constructed genetic nets. It seems reasonable to identify one cell
type with one state cycle. To the extent that this binary model, in which the
expression of the “gene” is potentially reversible at each clocked moment, is
accurate, it demonstrates the common occurrence of multiple modes of
behavior in a genetic system.
Tf this identification is reason

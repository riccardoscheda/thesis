\documentclass[12pt,a4paper]{report}
\usepackage[english]{babel}
%\usepackage{newlfont}

%%\usepackage{amsfonts}
\usepackage{amssymb}
\usepackage{hyperref}
\usepackage{bm}
\usepackage{graphicx}
\usepackage{color}
\usepackage{tikz}
\usetikzlibrary{arrows}

\usepackage{pdfpages}

\usepackage{bm}
\usepackage{newlfont}
\usepackage{color}
\textwidth=450pt\oddsidemargin=0pt
\usepackage{systeme}

\newcommand{\facciatabianca}{\newpage\shipout\null\stepcounter{page}}

\newcommand{\xtodo}[2][]{\tikzexternaldisable\todo[#1]{#2}\tikzexternalenable}
\usepackage{fancyhdr}

\usepackage{indentfirst}
\usepackage{graphicx}
%\usepackage{newlfont}

\usepackage{amssymb}
%\usepackage{amsmath}
\usepackage{latexsym}
\usepackage{amsthm}

\usepackage{palatino}
\usepackage{mathpazo} 
\usepackage[scaled=.95]{helvet} 
\usepackage{courier}
%
%\oddsidemargin=30pt \evensidemargin=20pt%impostano i margini
\hyphenation{}                          %serve per la sillabazione
\theoremstyle{plain}                    %stile corsivo
\newtheorem{teo}{Teorema}[section]      %definizione ambiente teorema
\newtheorem{prop}[teo]{Proposizione}    %definizione ambiente proposizione
\newtheorem{cor}[teo]{Corollario}       %definizione ambiente corollario
\newtheorem{lem}[teo]{Lemma}            %definizione ambiente lemma
\theoremstyle{definition}               %stile roman
\newtheorem{defin}{Definizione}[chapter]%definizione ambiente definizione
\newtheorem{ese}{Esempio}[chapter]      %definizione ambiente esempio
\theoremstyle{remark}                   %stile per osservazioni
\newtheorem{oss}{Osservazione}          %definizione ambiente osservazione
%%%%%%%%%%%%%%%%%%%%%%%%%%%%%%%%%%%%%%%%%comandi per l'impostazione
                                        %   della pagina, vedi il manuale
                                        %   della libreria fancyhdr
                                        %   per ulteriori delucidazioni
\pagestyle{fancy}\addtolength{\headwidth}{20pt}
\renewcommand{\chaptermark}[1]{\markboth{\thechapter.\ #1}{}}
\renewcommand{\sectionmark}[1]{\markright{\thesection \ #1}{}}
\rhead[\fancyplain{}{\bfseries\leftmark}]{\fancyplain{}{\bfseries\thepage}}
\cfoot{}
%%%%%%%%%%%%%%%%%%%%%%%%%%%%%%%%%%%%%%%%%
\linespread{1.3}                        %comando per impostare l'interlinea
%%%%%%%%%%%%%%%%%%%%%%%%%%%%%%%%%%%%%%%%%definisce nuovi comandi
\newcommand{\df}{\displaystyle\frac}    %crea un comando che visualizza le
                                        %   frazioni in modo più esteso
\newcommand{\seq}[1]{\left<#1\right>}   %crea un comando per il "generato"
                                        %   di un insieme, per richiamarlo
                                        %   si può scrivere ad esempio:
                                        %           $\seq{q_1,q_2}$
\begin{document}


\begin{titlepage}
\begin{center}
{{\Large{\textsc{Alma Mater Studiorum $\cdot$ University of  Bologna}}}}
\rule[0.1cm]{15.8cm}{0.1mm}
\rule[0.5cm]{15.8cm}{0.6mm}
\\\vspace{3mm}
{\small{\bf School of Science \\
Department of Physics and Astronomy\\
Master Degree in Physics}}
\end{center}

\vspace{23mm}

\begin{center}
    \LARGE{\bf Modeling cell differentiation using dynamical systems on graphs}\\
\end{center}

\vspace{50mm} \par \noindent

\begin{minipage}[t]{0.47\textwidth}
{\large{\bf Supervisor: \vspace{2mm}\\
Prof. Armando Bazzani}\\\\}
\end{minipage}
%
\hfill
%
\begin{minipage}[t]{0.47\textwidth}\raggedleft
    \textcolor{black}{
        {\large{\bf Submitted by:
            \vspace{2mm}\\
            {Riccardo Scheda}}}
    }
\end{minipage}

\vspace{30mm}

\begin{center}
Academic Year 2019/2020
\end{center}
\end{titlepage}

\facciatabianca

\chapter*{Abstract}

Real living cell is a complex system governed by many process which are not yet understood: the process of cell differentiation is one of these. 
Cell  differentiation is the process of cells in which cells of a specific type reproduces themselves and give raise to different type of cells.
Cell differentiation is governed by the so called Gene Regulatory Networks (GRNs).
A GRN is a collection of molecular regulators that interact with each other and with other substances in the cell to govern the gene expression levels of mRNA and proteins. 
Kauffman proposed for the first time in 1969 to model GRN through the so called Random Boolean Networks (RBN).
RBNs are networks in which each node can have only two possible values: 0 or 1, where each node represent a gene in GRN which can be "on" or "off".
These networks can model GRNs because the activity of one node represents the expression level of one gene among the whole regulation.

In this thesis work we make use of a mathematical model to develop gene reproduce a posbbile Gene Regulatory Network for the T-cells of the immune system, which present a bistable nature.



%%%%%%%%%%%%%%%%%%%%%%%%%%%%%%%%%%%%%%%%%non numera l'ultima pagina sinistra
\clearpage{\pagestyle{empty}\cleardoublepage}

\tableofcontents                        %crea l'indice
%%%%%%%%%%%%%%%%%%%%%%%%%%%%%%%%%%%%%%%%%imposta l'intestazione di pagina
\rhead[\fancyplain{}{\bfseries\leftmark}]{\fancyplain{}{\bfseries\thepage}}
\lhead[\fancyplain{}{\bfseries\thepage}]{\fancyplain{}{\bfseries
INDICE}}

%%%%%%%%%%%%%%%%%%%%%%%%%%%%%%%%%%%%%%%%%non numera l'ultima pagina sinistra
\clearpage{\pagestyle{empty}\cleardoublepage}
\listoffigures                          %crea l'elenco delle figure
%%%%%%%%%%%%%%%%%%%%%%%%%%%%%%%%%%%%%%%%%non numera l'ultima pagina sinistra



\chapter*{Introduction}                 %crea l'introduzione (un capitolo
                                        %   non numerato)
%%%%%%%%%%%%%%%%%%%%%%%%%%%%%%%%%%%%%%%%%imposta l'intestazione di pagina
\rhead[\fancyplain{}{\bfseries
INTRODUCTION}]{\fancyplain{}{\bfseries\thepage}}
\lhead[\fancyplain{}{\bfseries\thepage}]{\fancyplain{}{\bfseries
INTRODUCTION}}
%%%%%%%%%%%%%%%%%%%%%%%%%%%%%%%%%%%%%%%%%aggiunge la voce Introduzione
                                        %   nell'indice
\addcontentsline{toc}{chapter}{Introduction}



\chapter{Cell Differentiation }\label{celldiff}
\lhead[\fancyplain{}{\bfseries\thepage}]{\fancyplain{}{\bfseries\rightmark}}


\section{Gene Regulatory Networks}

All steps of gene expression can be modulated, since passage of the transcription of DNA to RNA, to the post-translational modification of the protein
produced. Hence, gene expression is a complex process regulated at several stages in the synthesis of proteins. In addition to the DNA transcription regulation, the expression of a gene may be controlled during RNA processing and transport (in eukaryotes), RNA translation, and the post-translational modification of proteins. This gives rise to genetic regulatory systems structured by networks of regulatory interactions between DNA, RNA, proteins and other molecules [6]: a complex network termed as a gene regulatory
network (GRN). Some, noteworthy, kind of proteins are the transcription factors that bind to specific DNA sequences in order to regulate the
expression of a given gene. The power of transcription factors resides in their ability to activate and/or repress transcription of genes. The activation of
a gene is also referred to positive regulation, while the negative regulation
identifies the inhibition of the gene.
The regulation of gene expression is essential for the cell, because it
allows to control the internal and external functions of the cell. Furthermore,
in multicellular organisms, gene regulation drives the processes of cellular
differentiation and morphogenesis, leading to the creation of different cell
types that possess different gene expression profiles, and these last therefore
produce different proteins that have different ultrastructures that suit them
to their functions (though they all possess the genotype, which follows the
same genome sequence) 4 . Therefore, with few exceptions, all cells in an
organism contain the same genetic material [6], and hence the same genome
(the haploid set of chromosomes of a cell). The difference between the cells
are emergent and due to regulatory mechanisms which can turn on or off
genes. Two cells are different, if they have different subsets of active genes.

\section{Cell Differentiation}
Cell differentiation is the process whereby stem cells become progressively
more specialized. The differentiation process occurs both during the devel-
opment of a multicellular organism and during tissue repair and cell turnover
in the adulthood. Gene expression, and therefore its regulatory mechanisms,
plays a critical role in cell differentiation; as described in the previous section.
Stem cells are undifferentiated biological cells which can both reproduce
themselves, self-renewal ability, and differentiate into specialized cells, po-
tency.

The principles underlying cellular differentiation remain among the most
enigmatic in biology. We are required to explain the spontaneous generation of a multiplicity of cell types from the single zygote, to deduce a natural
tendency of a system to become increasingly heterogeneous, then to stop
differentiating.

Among the important characteristics of cell differentiation are: initiation
of change; stabilization of change after cessation of stimulus; the efficacy of
many substances, exogenous and endogenous, as inductive stimuli; a limit
of five or six as the number of cell types which may differentiate directly from
any cell type; progressive limitation in the number of developmental path-
ways open to any small region of the embryo; restricted periods during which
a cell is competent to respond to an inductive stimulus; the discreteness of
cell types, that is, the mutually exclusive constellations of properties by
which cells differ; a requirement for a minimal and preferably heterogeneous
cell mass to initiate differentiation in many instances, and to maintain it in
some; the occurrence of metaplasia between undifferentiated cell types, or
from an undifferentiated type to a specialized type, but the lack of metaplasia
(the isolation) between specialized cell types; and the cessation of differentia-
tion (Grobstein, 1959).
9).
I believe many aspects of differentiation to be deducible from the typical
behavior of randomly built genetic nets.
Cells are thought to differ due to differential expression of, rather than
structural loss of, the genes. Differential activity of the genes raises at least
two questions which are not always carefully distinguished: the capacity of
the genome to behave in more than one mode; and mechanisms which insure
the appropriate assignment of these modes to the proper cells. The second
presumes the first.
Randomly assembled nets of binary elements behave in a multiplicity
of
distinct modes. Different state cycles embodied in a net are isolated from
each other, for no state may be on two cycles. Thus, a multiplicity
of state
cycles, each a different temporal sequence of genetic activity, is to be expected
in randomly constructed genetic nets. It seems reasonable to identify one cell
type with one state cycle. To the extent that this binary model, in which the
expression of the “gene” is potentially reversible at each clocked moment, is
accurate, it demonstrates the common occurrence of multiple modes of
behavior in a genetic system.
Tf this identification is reason

\chapter{Immunosystem and T-Cells}\label{tcells}
\lhead[\fancyplain{}{\bfseries\thepage}]{\fancyplain{}{\bfseries\rightmark}}

In this Chapter we briefly explain what T-cells are and why they are important for immune system.

\section{T-Cells}

\emph{T-cell}, also called \emph{T lymphocyte}, is a type of leukocyte (white blood cell) that is an essential part of the immune system. T-cells are one of two primary types of lymphocytes (B cells being the second type) that determine the specificity of immune response to antigens (foreign substances) in the body \cite{K39}.
T-cells originate in the bone marrow and mature in the thymus. In the thymus, T-cells multiply and differentiate into different type of cells: \emph{T helper}, \emph{regulatory T-cells} and \emph{cytotoxic T-cells}; further, during their life they can become \emph{memory T-cells}. They are then sent to peripheral tissues or circulate in the blood or lymphatic system. 
In short words, their main roles are:
\begin{itemize}
\centering
\item \textbf{T helper}: once stimulated by the appropriate antigen, helper T-cells secrete chemical messengers called \emph{cytokines}, which stimulate the differentiation of B cells into plasma cells (antibody-producing cells). 

\item \textbf{Regulatory T-cells}: act to control immune reactions, hence their name. 

\item \textbf{Cytotoxic T-cells}: they are activated by various cytokines, bind to and kill infected cells and cancer cells.
\end{itemize} 



Because the body contains millions of T and B cells, many of which carry unique receptors, it can respond to virtually any antigen.

Despite the structural similarities, the receptors on T-cells function differently from those on B cells. The functional difference underlies the different roles played by B and T-cells in the immune system. B cells secrete antibodies to antigens in blood and other body fluids, but T-cells cannot bind to free-floating antigens. Instead they bind to fragments of foreign proteins that are displayed on the surface of body cells. Thus, once a virus succeeds in infecting a cell, it is removed from the reach of circulating antibodies only to become susceptible to the defense system of the T-cell.

Some T-cells recognize class I MHC molecules on the surface of cells; others bind to class II molecules. Cytotoxic T-cells destroy body cells that pose a threat to the individual—namely, cancer cells and cells containing harmful microorganisms. Helper T-cells do not directly kill other cells but instead help activate other white blood cells (lymphocytes and macrophages), primarily by secreting a variety of cytokines that mediate changes in other cells. The function of regulatory T-cells is poorly understood. To carry out their roles, helper T-cells recognize foreign antigens in association with class II MHC molecules on the surfaces of macrophages or B cells. Cytotoxic T-cells and regulatory T-cells generally recognize targeT-cells bearing antigens associated with class I molecules. Because they recognize the same class of MHC molecule, cytotoxic and regulatory T-cells are often grouped together; however, populations of both types of cells associated with class II molecules have been reported. Cytotoxic T-cells can bind to virtually any cell in the body that has been invaded by a pathogen.

T-cells have another receptor, or coreceptor, on their surface that binds to the MHC molecule and provides additional strength to the bond between the T-cell and the targeT-cell. Helper T-cells display a coreceptor called CD4, which binds to class II MHC molecules, and cytotoxic T-cells have on their surfaces the coreceptor CD8, which recognizes class I MHC molecules. These accessory receptors add strength to the bond between the T-cell and the targeT-cell.
The T-cell receptor is associated with a group of molecules called the CD3 complex, or simply CD3, which is also necessary for T-cell activation. These molecules are agents that help transduce, or convert, the extracellular binding of the antigen and receptor into internal cellular signals; thus, they are called signal transducers. Similar signal transducing molecules are associated with B-cell receptors.

\section{Life cycle of T lymphocytes}
When T-cell precursors leave the bone marrow on their way to mature in the thymus, they do not yet express receptors for antigens and thus are indifferent to stimulation by them. Within the thymus the T-cells multiply many times as they pass through a meshwork of thymus cells. In the course of multiplication they acquire antigen receptors and differentiate into helper or cytotoxic T-cells. As mentioned in the previous section, these cell types, similar in appearance, can be distinguished by their function and by the presence of the special surface proteins, CD4 and CD8. Most T-cells that multiply in the thymus also die there. This seems wasteful until it is remembered that the random generation of different antigen receptors yields a large proportion of receptors that recognize self antigens (i.e. molecules present on the body's own constituents) and that mature lymphocytes with such receptors would attack the body’s own tissues.

Most such self-reactive T-cells die before they leave the thymus, so that those T-cells that do emerge are the ones capable of recognizing foreign antigens. These travel via the blood to the lymphoid tissues, where, if suitably stimulated, they can again multiply and take part in immune reactions. The generation of T-cells in the thymus is an ongoing process in young animals. In humans large numbers of T-cells are produced before birth, but production gradually slows down during adulthood and is much diminished in old age, by which time the thymus has become small and partly atrophied. Cell-mediated immunity persists throughout life, however, because some of the T-cells that have emerged from the thymus continue to divide and function for a very long time.

\section{Activation of T lymphocytes}
Helper T-cells do not directly kill infected cells, as cytotoxic T-cells do. Instead they help activate cytotoxic T-cells and macrophages to attack infected cells, or they stimulate B cells to secrete antibodies. Helper T-cells become activated by interacting with antigen-presenting cells, such as macrophages. Antigen-presenting cells ingest a microbe, partially degrade it, and export fragments of the microbe (i.e. antigens) to the cell surface, where they are presented in association with class II MHC molecules. A receptor on the surface of the helper T-cell then binds to the MHC-antigen complex. But this event alone does not activate the helper T-cell. Another signal is required, and it is provided in one of two ways: either through stimulation by a cytokine or through a costimulatory reaction between the signaling protein, B7, found on the surface of the antigen-presenting cell, and the receptor protein, CD28, on the surface of the helper T-cell. If the first signal and one of the second signals are received, the helper T-cell becomes activated to proliferate and to stimulate the appropriate immune cell. If only the first signal is received, the T-cell may be rendered anergic, that is, unable to respond to antigen.

Once the initial steps of activation have occurred, helper T-cells synthesize other proteins, such as signaling proteins and the cell-surface receptors to which the signaling proteins bind. These signaling molecules play a critical role not only in activating the particular helper T-cell but also in determining the ultimate functional role and final differentiation state of that T-cell. For example, the helper T-cell produces and displays IL-2 receptors on its surface and also secretes IL-2 molecules, which bind to these receptors and stimulate the helper T-cell to grow and divide.
The overall result of helper-T-cell activation is an increase in the number of helper T-cells that recognize a specific foreign antigen, and several T-cell cytokines are produced. The cytokines have other consequences, one of which is that IL-2 allows cytotoxic or regulatory T-cells that recognize the same antigen to become activated and to multiply. Cytotoxic T-cells, in turn, can attack and kill other cells that express the foreign antigen in association with class I MHC molecules, which (as explained above) are present on almost all cells. So, for example, cytotoxic T-cells can attack target T-cells that express antigens made by viruses or bacteria growing within them. Regulatory T-cells may be similar to cytotoxic T-cells, but they are detected by their ability to suppress the action of B cells or even of helper T-cells (perhaps by killing them). Regulatory T-cells thus act to damp down the immune response and can sometimes predominate so as to suppress it completely.

\section{Immunotherapy}
Early attempts to harness the immune system to fight cancer involved tumour-associated antigens, proteins that are present on the outer surface of tumour cells. Antigens are recognized as “foreign” by circulating immune cells and thereby trigger an immune response. However, many tumour antigens are altered forms of proteins found naturally on the surface of normal cells; in addition, those antigens are not specific to a certain type of tumour but are seen in a variety of cancers. Despite the lack of tumour specificity, some tumour-associated antigens can serve as targets of attack by components of the immune system. For instance, antibodies can be produced that recognize a specific tumour antigen, and those antibodies can be linked to a variety of compounds (such as chemotherapeutic drugs and radioactive isotopes) that damage cancer cells. In this way the antibody delivers the therapeutic agent directly to the tumour cell. In other cases a chemotherapeutic agent attached to an antibody destroys cancer cells by interacting with receptors on their surfaces that trigger apoptosis.

T-cells themselves may be engineered to recognize, bind to, and kill cancer cells. For example, in an experimental treatment for chronic lymphocytic leukemia, researchers designed a virus to induce the expression on patient T-cells of antibody receptors that identified and attached to antigens on malignant B cells and that activated the T-cells, prompting them to destroy the B cells. T-cells removed from patient blood were incubated with the virus and following infection were infused back into the patient. A portion of the engineered cells persisted as memory T-cells, retaining functionality and suggesting that the cells possessed long-term activity against cancer cells.

A similar T-cell therapy, known as \emph{chimeric antigen receptor T-cells} (CAR-T), in which T-cells isolated from a patient’s blood are genetically engineered to specifically identify and target cancer cells and then are infused back into the patient, has been used in the treatment of certain forms of leukemia, including acute lymphocytic leukemia, as well as B-cell lymphoma. The addition, via genetic engineering, of a unique receptor to the T-cell surface that is capable of recognizing a molecule known as MR1, found on cells from a variety of different cancer types, has opened the possibility of expanding CAR-T to the treatment of solid tumours, in addition to cancers of the blood.



\chapter{Gene Regulatory Networks}\label{grn}
\lhead[\fancyplain{}{\bfseries\thepage}]{\fancyplain{}{\bfseries\rightmark}}


\section{Definition}

Gene regulation controls the expression of genes and, consequently, all cellular functions. Gene expression is a process that involves transcription of the gene into mRNA, followed by translation to a protein, which may be subject to post-translational modification \cite{K40}. The transcription process is controlled by transcription factors (TFs) that can work as activators or inhibitors. TFs are themselves encoded by genes and subject to regulation, which altogether forms complex regulatory networks. 
\begin{figure}
\centering
\includegraphics{GRN.png}
\end{figure}
Cells efficiently carry out molecular synthesis, energy transduction, and signal processing across a range of environmental conditions by networks of genes, which we define broadly as networks of interacting genes, proteins, and metabolites \cite{K37}. Formally speaking, a gene regulatory network or genetic regulatory network (GRN) is a collection of DNA segments in a cell which interact with each other (indirectly through their RNA and protein expression products) and with other substances in the cell, thereby governing the rates at which genes in the network are transcribed into mRNA. In general, each mRNA molecule goes on to make a specific protein (or set of proteins). In some cases this protein will be structural, and will accumulate at the cell-wall or within the cell to give it particular structural properties. 



\section{Role within the cell}

Gene Regulatory Networks (GRNs) control biological process of all organisms. The complex control systems underlying development have probably been evolving for more than a billion years. They regulare the expression of thousand of genes in any given biological process. They are essentially hardwired genomic regulatory codes, the role of which is to speicify the sets of genes that must be expressed in specific spatial and temporal patterns. In physical terms, these control system consist of many thousands of modular DNA sequences. Each module receives and integrates multiple inputs, in the form of regulatory proteins (activators and repressors) that recognize specific sequences within them. The end result is the precise transcriptional control of the associated genes. 
Functional linkages between these particular genes, and their associated regulatory modules, define the core networks underlying development. They explain exactly how genomic sequence encodes the regulation of expression of the sets of genes that generate patterns and execute the construction of multiple states of differentiation.

The regulatory genome is a logic processing system: every regulatory module contained in the genome receives multiple inputs and processes in ways that can be mathematically represented as combinations of logic functions. 

Definitive regulatory functions emerge only from the architecture of intergenic linkages, and these functions are not visible at the level of any individual genes. So gene regulatory networks can be determined only by experimental molecular biology in which the functional meaning of given regulatory sequences is directly determined.

GRNs have a complex structure: they are inhomogeneous compositions of different kinds of subnetworks, each performing a specific kind of function. Some subnetworks are used in many processes.
\section{Bo}
In recent years, single-cell-resolution measurements have revealed unprecedented levels of cell-to-cell heterogeneity within tissues. The discovery of this ever-present heterogeneity is driving a more nuanced view of cell phenotype, wherein cells exist along a continuum of cell-states, rather than conforming to discrete classifications. The comprehensive view of diverse cell states revealed by single cell measurements is also affording new opportunities to discover molecular regulators of cell phenotype and dynamics of lineage commitment (Trapnell et al., 2014; Olsson et al., 2016; Briggs et al., 2018). For example, single cell transcriptomics have revealed the widespread nature of multilineage priming (MLP), a phenomenon wherein individual, multipotent cells exhibit “promiscuous” coexpression of genes associated with distinct lineages prior to commitment (Nimmo et al., 2015). In principle, mathematical modeling of gene regulatory network dynamics can provide a theoretical foundation for understanding cell heterogeneity and gene expression dynamics, by quantitatively linking molecular-level regulatory mechanisms with observed cell states. However, due to the molecular complexity of gene regulatory mechanisms, it remains challenging to integrate such models with single-cell data.

Mathematical models of gene regulatory network dynamics can account for (and at least partially reproduce) observed cellular heterogeneity in two primary ways. First, gene network models are multistable dynamical systems, meaning a given network has the potential to reach multiple stable states of gene expression. These states arise from the dynamic interplay of activation, inhibition, feedback, and nonlinearity (Kauffman, 1969; MacArthur et al., 2009; Huang, 2012). Second, some mathematical models inherently treat cellular noise. This noise, or stochasticity, is modeled in various ways depending on assumptions about the source (Peccoud and Ycart, 1995; Arkin et al., 1998; Kepler and Elston, 2001; Swain et al., 2002). Discrete, stochastic models of gene regulation, which track discrete molecular entities, regulatory-protein binding kinetics, and binding states of promoters controlling gene activity, have formed the basis of biophysical theories of gene expression noise due to so-called intrinsic molecular noise (Peccoud and Ycart, 1995; Thattai and van Oudenaarden, 2001; Kepler and Elston, 2001; Pedraza and Paulsson, 2008). Such stochastic gene-regulation mechanisms have also been incorporated into larger regulatory network models using the formalism of stochastic biochemical reaction networks, and have been utilized to explore how molecular fluctuations can cause heterogeneity within phenotype-states and promote stochastic transitions between phenotypes (Feng and Wang, 2012; Sasai et al., 2013; Zhang and Wolynes, 2014; Tse et al., 2015).

The quantitative landscape of cellular states is another concept that is increasingly utilized to describe cellular heterogeneity. Broadly, the cellular potential landscape (first conceptualized by Waddington (Wang et al., 2011; Huang, 2012; Waddington, 2014) is a function in high-dimensional space (over many molecular observables, typically expression levels of different genes), that quantifies the stability of a given cell-state. In analogy to potential energy (gravitational, chemical, electric, etc.), cell states of higher potential are less stable than those of lower potential. The landscape concept inherently accounts for cellular heterogeneity, since it holds that a continuum of states is theoretically accessible to the cell, with low-potential states (in “valleys”) more likely to be observed than high-potential states. The landscape is a rigorously defined function derived from the dynamics of the underlying gene network model, according to some choice of mathematical formalism (Wang et al., 2011; Bhattacharya et al., 2011; Huang, 2012; Zhou et al., 2016). For stochastic gene network models that inherently treat noise, the landscape is directly obtained from the computed probability distribution over cell-states (Cao and Liang, 2008; Micheelsen et al., 2010; Feng and Wang, 2012; Tse et al., 2015).

Stochastic modeling of gene network dynamics has been employed in various forms for analysis of single cell measurements. For example, application of noisy dynamical systems theory has shed light on cell-state transitions (Mojtahedi et al., 2016; Jin et al., 2018; Lin et al., 2018). Stochastic simulations of gene network dynamics have been used to develop and/or benchmark tools for network reconstruction (Schaffter et al., 2011; Dibaeinia and Sinha, 2019; Bonnaffoux et al., 2019) Stochastic model-aided analysis of single-cell measurements has been demonstrated to yield insights on gene regulatory mechanisms (Munsky et al., 2018). However, few existing analysis methods utilize discrete-molecule, stochastic models, which fully account for intrinsic gene expression noise and its impact on cell-state, to aid in the interpretation of noisy distributions recovered from single cell RNA sequencing data. There exists an opportunity to link such biophysical, stochastic models, which reproduce intrinsic noise and cell heterogeneity in silico, to single cell datasets that characterize cell heterogeneity in vivo. In particular, the landscape of heterogeneous cell-states computed from discrete stochastic models can be directly compared to single-cell measurements.


\section{GRN}
The cells of living organisms di erentiate within the developing embryo into
the various cell types that form tissues by a process that is regulated at the
molecular level by DNA sequences, encoding genes that produce proteins that
regulate other genes. All eukaryotic cells in an organism carry an identical set
of genes, some of which are expressed others not. A cell type is de ned by the
particular subset of genes that are expressed. The gene expression pattern of
a cell needs to be stable but also adaptable \cite{K41}.



Genes regulate each other's activity by coding for transcription factors,
which may enhance or repress the expression of other genes by binding (pos-
sibly in combination) at particular sites. Though a particular gene directly
regulates just a small set of other genes, those genes regulate other genes in
turn, so a gene will indirectly in uence the activity of many genes downstream.
Conversely, a particular gene is indirectly in uenced by many genes upstream.
A gene may directly or indirectly contribute to regulating itself. The result is
a genomic regulatory network, a complex feedback web of genes turning each
other on and o . This may be interpreted as an idealized dynamical system of
model genes with directional links (transcription factors), updating
their state in parallel, according to the combinatorial logic of their
inputs, Kauffman's Random Boolean Networks \cite{K1}{K7}.
There is justified debate as to whether parallel (synchronous) updating,
and the on-off characterization of genes, are valid idealizations when applied
to real genomic networks, given that transcription is asynchronous and driven
at different rates. However, gene activity at the molecular scale consists of
discrete events occurring concurrently. Variable protein concentrations can be
accounted for by genes being on for some fraction of a given time span. The
RBN idealization is arguably a valid starting point for gaining insights into
gene network dynamics.
In a cell type's gene expression pattern over a span of time (i.e. its space-
time pattern), a particular gene may, broadly speaking, be either on, off, or
changing. If a large proportion of the genes are changing, chaotic dynamics, the cell will be unstable. On the other hand, dynamics that settles to
a pattern where a large proportion of the genes are permanently on or o
(frozen) may be too in exible for adaptive behavior. Cells constantly need to
adapt their gene expression pattern in response to a variety of hormone and
growth/di erentiation factors from nearby cells. The de nition of a cell type
may be more correctly expressed as a set of closely related gene expression
patterns, allowing an essential measure of exibility in behavior.
\section{Gene Regulatory Networks}

All steps of gene expression can be modulated, since passage of the transcription of DNA to RNA, to the post-translational modification of the protein
produced. Hence, gene expression is a complex process regulated at several stages in the synthesis of proteins. In addition to the DNA transcription regulation, the expression of a gene may be controlled during RNA processing and transport (in eukaryotes), RNA translation, and the post-translational modification of proteins. This gives rise to genetic regulatory systems structured by networks of regulatory interactions between DNA, RNA, proteins and other molecules [6]: a complex network termed as a \emph{gene regulatory
network} (GRN). Some kind of proteins are the transcription factors that bind to specific DNA sequences in order to regulate the
expression of a given gene. The power of transcription factors resides in their ability to activate and/or repress transcription of genes. The activation of
a gene is also referred to positive regulation, while the negative regulation
identifies the inhibition of the gene.
The regulation of gene expression is essential for the cell, because it
allows to control the internal and external functions of the cell. Furthermore,
in multicellular organisms, gene regulation drives the processes of cellular
differentiation and morphogenesis, leading to the creation of different cell
types that possess different gene expression profiles, and these last therefore
produce different proteins that have different ultrastructures that suit them
to their functions (though they all possess the genotype, which follows the
same genome sequence) 4. Therefore, with few exceptions, all cells in an
organism contain the same genetic material [6], and hence the same genome. The difference between the cells are emergent and due to regulatory mechanisms which can turn on or off genes. Two cells are different, if they have different subsets of active genes.

\chapter{Random Boolean Networks}
\lhead[\fancyplain{}{\bfseries\thepage}]{\fancyplain{}{\bfseries\rightmark}}


In this chapter we explain the basic concepts of Random Boolean Network proposed for the first time by Kauffman.

\section{Random Boolean Networks}
Random Boolean networks (RBNs) were introduced in
1969 by S. Kauffman as a simple model of genetic systems.
Each gene was represented by a node
that has two possible states, “on” (corresponding to a
gene that is being transcribed) and “off” (corresponding
to a gene that is not being transcribed). There are altogether N nodes, and each node receives input from K
randomly chosen nodes, which represent the genes that
control the considered gene. Furthermore, each node is
assigned an update function that prescribes the state of
the node in the next time step, given the state of its input nodes. This update function is chosen from the set
of all possible update functions according to some probability distribution. Starting from some initial configuration, the states of all nodes of the network are updated
in parallel. Since configuration space is finite and since
dynamics is deterministic, the system must eventually return to a configuration that it has had before, and from
then on it repeats the same sequence of configurations
periodically: it is on an attractor.

\section{The model}
Let's consider a network of $N$ nodes. The state of each node at a time $t$ is given by $\sigma_i(t) \in {0,1}$ with $ i = 1,...,N$.
The $N$ nodes of the network can therefore together assume $2^N$ different states.
The number of incoming links to each node $i$  is denoted by $k_i$ and is drawn
randomly independently from the distribution $P(k_i)$.
The dynamical state of each $\sigma_i(t)$ is updated synchronously by a Boolean function $\Lambda_i$:
$$
\Lambda_i:\{0,1\}^{k_i} \to \{0,1\}
$$ 
An update function specifies
the state of a node in the next time step, given the state
of its $K$ inputs at the present time step. Since each of the
$K$ inputs of a node can be on or off, there are $M = 2^K$ possible input states.
The update function has to specify the new state of a node for each of these input states.
Consequently, there are $2^M$ different update functions.
For example let's consider a network with $K=1$, so all the functions $\Lambda_i$ receives the input from one single node. 
In general each element 
receives inputs from exactly $K$ nodes, so we have a dynamical system defined from:

\begin{equation}
\sigma_i(t+1)=\Lambda_i(\sigma_{i_1}(t),\sigma_{i_2}(t), ...,\sigma_{i_K}(t)).
\end{equation}  

So, the randomness of these network appears at two levels: in the connectivity of the network (which node is linked
to which) and the dynamics (which function is attributed to which node).

\section{Topology}
For a given number $N$ of nodes and a given number
$K$ of inputs per node, a RBN is constructed by choosing
the $K$ inputs of each node at random among all nodes.
If we construct a sufficiently large number of networks in
this way, we generate an ensemble of networks. In this
ensemble, all possible topologies occur, but their statis-
tical weights are usually different. Let us consider the
simplest possible example, $N = 2$ and $K = 1$, shown
in Figure~\ref{fig:rb}. There are 3 possible topologies.



\begin{figure}[h]
\centering
\includegraphics[scale=1]{figurenetworks.pdf}
\caption{Set of all possible networks with $N=2$ and $K=1$.}
\label{fig:rb}
\end{figure}

Topologies (a) and (b) have each the statistical weight $1/4$ in
the ensemble, since each of the links is connected in the
given way with probability $1/2$. Topology (c) has the
weight $1/2$, since there are two possibilities for realizing
this topology: either of the two nodes can be the one
with the self-link.


While the number of inputs of each node is fixed by
the parameter $K$, the number of outputs (i.e. of outgo-
ing links) varies between the nodes. The mean number of
outputs must be $K$, since there must be in total the same
number of outputs as inputs. A given node becomes the
input of each of the N nodes with probability $\frac{K}{N}$ . In
the thermodynamic limit $N \to \infty$ the probability distribution of the number of outputs is therefore a Poisson
distribution:

$$
P_{out}(k) = \frac{K^k}{k!}e^{-K}
$$

\section{Dynamics}

All nodes are updated at the same time
according to the state of their inputs and to their update
function. Starting from some initial state, the network
performs a trajectory in state space and eventually arrives on an \emph{attractor}, where the same sequence of states
is periodically repeated. Since the update rule is deterministic, the same state must always be followed by the
same next state. If we represent the network states by
points in the $2^N$-dimensional state space, each of these
points has exactly one “output”, which is the successor
state. We thus obtain a graph in state space.
The size or length of an attractor is the number of
different states on the attractor. The basin of attraction
of an attractor is the set of all states that eventually
end up on this attractor, including the attractor states
themselves. The size of the basin of attraction is the
number of states belonging to it. The graph of states
in state space consists of unconnected components, each
of them being a basin of attraction and containing an
attractor, which is a loop in state space. The transient
states are those that do not lie on an attractor. They are
on trees leading to the attractors.


Let us illustrate these concepts by studying the small
$K = 1$ network shown in Figure~\ref{fig:rb2}, which consists of 4
nodes:

\begin{figure}[h]
\centering
\includegraphics[scale=1]{figurenetworks2.pdf}
\caption{A small network with $N=4$ and $K=1$.}
\label{fig:rb2}
\end{figure}

If we assign to the nodes 1,2,3,4 the functions invert,
invert, copy, copy, an initial state $1111$ evolves in the
following way:

$$
1111 \to 0011 \to 0100 \to 1111
$$

This is an attractor of period 3. If we interpret the bit se-
quence characterizing the state of the network as a number in binary notation, the sequence of states can also be
written as

$$
15 \to 3 \to 4 \to 15
$$

The entire state space is shown in Figure~\ref{fig:rb3}:
\begin{figure}[h]
\centering
\includegraphics[scale=1]{figurenetworks2.pdf}
\caption{The state space of the network shown in Figure~\ref{fig:rb2}, if
the functions copy, copy, invert, invert are assigned to the four
nodes. The numbers in the squares represent states, and ar-
rows indicate the successor of each state. States on attractors
are shaded.}
\label{fig:rb3}
\end{figure}

There are 4 attractors, two of which are fixed points
(i.e., attractors of length 1). The sizes of the basins of
attraction of the 4 attractors are 6,6,2,2. If the function
of node 1 is a constant function, fixing the value of the
node at 1, the state of this node fixes the rest of the
network, and there is only one attractor, which is a fixed
point. Its basin of attraction is of size 16. If the functions
of the other nodes remain unchanged, the state space
then looks as shown in Figure~\ref{fig:rb4}

\begin{figure}[h]
\centering
\includegraphics[scale=1]{figurenetworks2.pdf}
\caption{The state space of the network shown in Figure~\ref{fig:rb2},
if the functions 1, copy, invert, invert are assigned to the four
nodes.}
\label{fig:rb4}
\end{figure}

Before we continue, we have to make the definition of
attractor more precise: as the name says, an attractor
“attracts” states to itself. A periodic sequence of states
(which we also call cycle) is an attractor if there are states
outside the attractor that lead to it. However, some netaworks contain cycles that cannot be reached from any
state that is not part of it. For instance, if we removed
node 4 from the network shown in Figure 2.2, the state
space would only contain the cycles shown in Figure 2 C,
and not the 8 states leading to the cycles. In the follow-
ing, we will use the word “cycle” whenever we cannot be
confident that the cycle is an attractor.


\section{Applications}

Let us now make use of the definitions and concepts
introduced in this section in order to derive some results
concerning cycles in state space. First, we prove that in
an ensemble of networks with update rule 1 (biased functions) or rule 2 (weighted classes), there is on an average
exactly one fixed point per network. A fixed point is a
cycle of length 1. The proof is slightly different for rule
1 and rule 2. Let us first choose rule 2. We make use of
the property that for every update function the inverted
function has the same probability. The inverted function
has all 1s in the output replaced with 0s, and vice versa.
Let us choose a network state, and let us determine for
which fraction of networks in the ensemble this state is a
fixed point. We choose a network at random, prepare it
in the chosen state, and perform one update step. The
probability that node 1 remains in the same state after
the update, is 1/2, because a network with the inverted
function at node 1 occurs equally often. The same holds
for all other nodes, so that the chosen state is a fixed
point of a given network with probability $2^{−N}$ . This
means that each of the 2 N states is a fixed point in the
proportion $2^{−N}$ of all networks, and therefore the mean
number of fixed points per network is 1. We will see
later that fixed points may be highly clustered: a small
proportion of all networks may have many fixed points,
while the majority of networks have no fixed point.


Next, we consider rule 1. We make now use of the
property that for every update function a function with
any permutation of the input states has the same probability. This means that networks in which state A leads
to state B after one update, and networks in which another state C leads to state B after one update, occur
equally often in the ensemble. Let us choose a network
state with n 1s and $N − n$ 0s. The average number of
states in a network leading to this state after one update
is $2^N p^n (1 − p^{N −n}$ . Now, every state leads equally often
to this state, and therefore this state is a fixed point in
the proportion $p^n (1 − p)^{N−n}$ of all networks. Summation
over all states gives the mean number of fixed points per
network, which is 1.

Finally, we derive a general expression for the mean
number of cycles of length L in networks with $K = 2$
inputs per node. The generalization to other values of $K$
is straightforward. Let $\langle C_L\rangle_N$ denote the mean number
of cycles in state space of length $L$, averaged over the
ensemble of networks of size $N$ . On a cycle of length
L, the state of each node goes through a sequence of
1s and 0s of period L. Let us number the $2^L$ possible
sequences of period L of the state of a node by the index
j, ranging from 0 to $m = 2^L − 1$. Let $n_j$ denote the
number of nodes that have the sequence $j$ on a cycle of
length$ L$, and $(P_L )^j_{l,k}$ the probability that a node that
has the input sequences $l$ and $k$ generates the output
sequence $j$. This probability depends on the probability distribution of update functions.



Then

\begin{equation}
\langle C_L\rangle_N = \frac{1}{L} \sum_{n_j} \frac{N!}{n_0! \dots n_m!} \prod_j \big(\sum_{l,k}\frac{n_ln_k}{N^2}(P_L)^j_{l,k}\big)^{n_j}
\end{equation}

The factor 1/L occurs because any of the L states on the
cycle could be the starting point. The sum is P
over all
possibilities to choose the values {n j } such that j n j =
N . The factor after the sum is the number of different
ways in which the nodes can be divided into groups of the
sizes n 0 , n 1 , n 2 , . . . , n m . The product is the probability
that each node with a sequence j is connected to nodes
with the sequences l and k and has an update function
that yields the output sequence j for the input sequences
l and k. This formula was first given in the beautiful
paper by Samuelsson and Troein [10].


%\chapter{Cancer attractors}
\lhead[\fancyplain{}{\bfseries\thepage}]{\fancyplain{}{\bfseries\rightmark}}

%\newcommand{\folder}{/path/to/folder}
\newenvironment{sistema}%
{\left\lbrace\begin{array}{@{}l@{}}}%
{\end{array}\right.}
In questo capitolo viene fatta una piccola introduzione al modello classico preda-predatore di Lotka-Volterra.
\section{The model}
We consider a physical system that can be described by an weighted interaction network among nodes that can assume different
dynamical states (in the case of a gene network the states $\sigma\in [0,1]$ and we have models similar to spin models).
In the simplest case, we introduce a stochastic dynamics using the probability $p_i(t)$ that the node $i$ is in the state $\sigma_i=1$
(then $1-p_i(t)$ is the probability to get $\sigma_i=0$) and we define a linear equation for the probability evolution
\begin{equation}
\dot p_i(t)=\sum_j \mathcal{P}_{ij}p_j(t)-\gamma_i p_i(t)
\label{average}
\end{equation}
where $\mathcal{P}_{ij}$ are transition probability rates and $\gamma_i^{-1}$ defines the mean lifetime of the excited state.
The meaning of the rates $\mathcal{P}_{ij}$ is the rate at which the excited state of the node $j$ increases (or decreases if
$\mathcal{P}_{ij}<0$) the probability of a transition to the excited state of the node $i$. Since $0\le p_i\le 1$ for all $i$, this space should be invariant for the dynamics. This condition depends on the spectral properties of the matrix
\begin{equation}
\mathcal{P}_{ij}-\gamma_j\delta_{ij}
\label{matrix}
\end{equation}
associated to the system. Let consider the case $\mathcal{L}_{ij}\ge 0$ (i.e. we have no inhibitory link),
the first quadrant is clearly invariant and if we define
$$
\sum_i  \mathcal{P}_{ij}=\hat \gamma_j>0
$$
the matrix 
$$
\mathcal{L}_{ij}=\mathcal{P}_{ij}-\hat \gamma_j \delta_{ij}
$$
is a Laplacian matrix and the system (\ref{average}) can be written in the form
$$
\dot p_i(t)=\sum_j \mathcal{L}_{ij}p_j(t)-\Delta \gamma_i p_i(t)\qquad \Delta \gamma_i=\gamma_i-\hat \gamma_i
$$
and by assumption we have $\gamma_i>\hat \gamma_i$. The eigenvalues of the matrix $\mathcal{L}_{ij}$
have all negative real part except the null eigenvalue. It follows that all the eigenvalue of the matrix (\ref{matrix})
has negative real part and the dynamics is a contraction towards the origin:
a stable solution (i.e. without any external stimulus the system relaxes to the $\sigma_i=0$ state).
A non trivial stationary can be achieved only if an external stimulus is inserted
\begin{equation}
\dot p_i(t)=\sum_j \mathcal{P}_{ij}p_j(t)-\gamma_i p_i(t)+\epsilon f_i(t)
\label{average_ext}
\end{equation}
The stationary solution has to satisfy $p_i\in [0,1]$ so that $f_i(t)\ge 0$ otherwise we can have negative probability 
when $p_i\simeq 0$. The case of a Laplacian matrix
$$
\hat \gamma_i=\gamma_i
$$
we get another possible stationary solution for $\mathcal{L}_{ij}p^\ast_j=0$ in the first quadrant and
the subspace $\sum p_i=0$ is invariant and the dynamics is a contraction in this subspace (in general).
Then the system a stable stationary solution even in absence of an external stimulus.\par\noindent
The presence of inhibitory links complicates the model and one has to prove that
\begin{itemize}
\item 1) there exists a physical space: an invariant cone in the first quadrant where the dynamics is a contraction towards
the origin;\par
\item 2) the external stimulus maintains the solution in the physical space.
\end{itemize}
Another solution could be to introduce boundary conditions so that $p_i\ge 0$ in any case (the system is non linear in such a case).
\par\noindent
The eigenvalues of the matrix (\ref{matrix}) define the different relaxation time scale the process and determine its rectivity
to the change of the external stimulus: in a typical problem one consider a slowly varying external stimulus so that the
system could be considered i a quasi stationary state
$$
\sum_j \mathcal{L}_{ij}p_j-\Delta \gamma_i p_i=-\epsilon f_i(t) \qquad \frac{df_i}{dt}\ll 1
$$
the derivative is small with respect to the eigenvalues of th matrix (adiabatic approximation). On the other hand we have the effect
of a correlated noise (we need to introduce a correlation in order have a continuous function $f_i(t)$). The problem is to
study the relation between the solution and the spectral properties of the matrix $\mathcal{L}_{ij}$: we simplify the
equation by assuming $\Delta \gamma_i=\Delta \gamma$ so that if $\lambda$ is an eigenvalue of $\mathcal{L}_{ij}$
then $\lambda-\Delta \gamma$ is an eigenvalue of the matrix (\ref{matrix}) and we assume that the dynamics
is perturbed by
\begin{equation}
\dot p_i(t)=\sum_j\left ( \mathcal{L}_{ij}+\Delta \mathcal{L}_{ij}\right ) p_j(t)-\Delta \gamma p_i(t)+\epsilon f_i(t)
\label{average_p}
\end{equation}
where the perturbation $\Delta\mathcal{L}_{ij}$ is a Laplacian matrix ($\sum_i \Delta\mathcal{L}_{ij}=0$ and we
assume $<\mathcal{L}>=0$) that can
represent an error in the measure of the transition rates $\mathcal{L}_{ij}$ or possible evolution of network due to
in time. In the first case we have an ensemble of transition matrices and we have to study the eigenvalue distribution
due to perturbation and the possible presence of bifurcation phenomena. In the second case we have a stochastic 
differential equation (since $\Delta \mathcal{L}_{ij}(t)$ can be represented as a realization of a stochastic process).
The possible approach are Perturbation Theory, Random Matrix Theory and Statistical Physics Methods for random matrices.
The external signal form the environment  (the environmental node) can be considered in the adiabatic approximation
(to be justified form a biological point of view).\par\noindent
The underlying stochastic process on the graph is defined by assigning the state $\xi_i(t)\in [0,1]$ at each node $i$ according 
to a probability distribution $\pi_i(t)$ that evolves as
$$
\dot \pi_i(t)=\sum_j \mathcal{L}_{ij}\xi_j(t)-\Delta \gamma \xi_i(t)+\epsilon f_i(t)
$$
By discretizing the dynamics for a time step $\Delta t$ we have the evolution
$$
\pi_i(t+\Delta t)=\pi(t)+\sum_j \mathcal{L}_{ij}\xi_j(t)\Delta t-\Delta \gamma \Delta t\xi_i(t)+\epsilon f_i(t)\Delta t
$$
and $\xi(t+\Delta t)$ realized according to the distribution $\pi_i(t+\Delta t)$ (stochastic cellular automata).
The average dynamics is computed by
\begin{eqnarray}
\dot <\pi_i(t)>&=&\sum_j \mathcal{L}_{ij}<\xi_j(t)>-\Delta \gamma <\xi_i(t)>+\epsilon f_i(t)\nonumber \\
&=&\sum_j \mathcal{L}_{ij}p_j(t)-\Delta \gamma p_i(t)+\epsilon f_i(t)=\dot p_i(t)\nonumber
\end{eqnarray}
and we recover the average equation (\ref{average}). But the stochastic dynamics gives information on the applicability
of the average approximation and the variability at the critical states (at bifurcation of the spectrum of $\mathcal{L}$).
The stochastic dynamics can be studied for stochastic connection matrices $\mathcal{L}+\Delta \mathcal{L}$.



%\documentclass[runningheads]{llncs}
%
%%\usepackage[latin9]{inputenc}
%%\usepackage{amsmath,amsthm}
%%\usepackage{braket}
%%\usepackage{amsfonts}

% Used for displaying a sample figure. If possible, figure files should
% be included in EPS format.
%
% If you use the hyperref package, please uncomment the following line
% to display URLs in blue roman font according to Springer's eBook style:
% \renewcommand\UrlFont{\color{blue}\rmfamily}
%
%
%\begin{document}
%
%\title{Stochastic models for dynamical systems on graphs and applications to genetic activity}
\chapter{The model}\label{model}
\lhead[\fancyplain{}{\bfseries\thepage}]{\fancyplain{}{\bfseries\rightmark}}
Models of boolean networks proposed by Kauffmann are limited and can't represent perfectly biological networks because of some considerations: simple RBNs present cahotic behaviours and attractors are not sufficient to explain cellular differentiation.
In this Chapter we present the theoretical model of GRNs based on RBNs.

\section{The underlying philosophy of the model}
The dynamical model is a schematic representation of the activity of gene regulatory networks introduced in Chapter \ref{grn}. We have to discuss the assumptions the define the model from a biological 
point of view: the main criticism to a model is that its assumptions cannot be justified by the biological mechanisms. 
Our goal is to model the genetic activity related to a differentiation process of a cell: i.e. this activity is a stable long term activity whose stability
is probably controlled by biochemical mechanisms (i.e. methylation processes), but for cancer cells the control dynamics is not so efficient allowing
the evolution of different cell populations. Then we assume that this evolution is possible due to the competition of different genetic activities through
dynamical mechanisms that can be triggered by the external environmental signals.
In particular we assume:
\begin{itemize}
\item the long term genetic activity is determined by the presence of small genetic networks that have a stable active dynamical state;
\item there exists an eternal control mechanism: the subnetworks have control nodes that prevent the arise of the active state in the subnetwork if there
are set to the inactive state;
\item once the active state has been established in a subnetwork it remains stable in time without any stimulus, except if an inhibitory stimulus 
change the state of control nodes;
\item the stability and the controllability properties of a subnetwork depends from the existence of loops in the subnetwork: a loop may be related to the activation
of metabolic cycles in the cell that define the cell behavior;
\item each node of a subnetwork may represent the state of a gene that is connected and regulates the activation of other genes;
\item in the cell differentiation mechanism is defined by the competition of different subnetworks that interact in a inhibitory way;
\item the mutation mechanism change the connectivity of the network: we may distinguish between permanent changes and dynamical change (i.e. a connection
may exist or non exist during time).
\end{itemize}
The complexity of the model is not a fundamental issue since we want to point out universal behaviors:  first of all the existence of bistability or bifurcation 
phenomena for simple model and the definition of control parameters.
%
\section{The mathematical model and related problems}
Here we studied the model dynamics in different situations using mathematical methods. The main idea is understand the dynamics of the models to point
out the universal properties that are robust and could explain the experimental data. The biological meaning of control parameters is a fundamental task
to apply the model to predict the results of new experiments.\par\noindent 
We consider a physical system that can be described by an weighted interaction network among nodes that can assume different
dynamical states (in the case of a gene network the states $\sigma\in \{0,1\}$ and we have models similar to spin models).
The interaction structure is defined by signed adjacency matrix $A_{ij}\in\{-1,0,1\}$ where the sign refers to a cooperative or antagonist interaction between the connected nodes. 
In the simplest case, we introduce a stochastic dynamics using the probability $p_i(\sigma, t)$ that the node $i$ is in the state 
$\sigma$ (we assume $\sigma>0$) at time $t$: in a deterministic approach $p_i(\sigma, t)=\delta(\sigma-\sigma(t))$
to denote that the node assume the state $\sigma=1$ with probability one. The evolution of a deterministic model
can be described by the equation
\begin{equation}
\sigma_i(t+1)=\Phi_i(\sigma(t))=\Theta\left (\sum_j A_{ij}\sigma_j(t)\right )
\label{evolnet}
\end{equation}
where $\Theta(x)\in[0,1]$ is a threshold sigmoidal function (we assume $A_{ii}=0$ to avoid self loops). \par\noindent
Remark: the dynamics is a information diffusion on the network. If we consider the linear system
$$
\zeta_i(t+1)=\sum_j A_{ij}\zeta_j(t)
$$
where $\zeta_i$ are non negative integers we have an equivalent dynamics since $\sigma_i=\Theta(\zeta_i)$ and it is possible
to study the linear system to derive some properties of the initial system. For example the relaxation time to the solution $\sigma_i=1$
$\forall\; i$ is for a given initial condition $\sigma^0_j=\delta_{jk}$ is $t=n$ such that the matrix $A^n$ has positive entries along the whole
$k$-th column. This mens that for each node $i$ there is a walk of length $n$ from the initial node $k$ to $i$.
\par\noindent
We also assume a cause-effect relation so that $A_{ij}$ is a directed
graph. The deterministic model is a Hopfield network (each node has at least an input and an output link; the environment nodes has only output links)
and one could study the equilibrium states and their stability. An equilibrium condition as follows is characterized as follows:
for each $i$ let
$$
Q_i(t)=\sum_j A_{ij}\sigma_j(t) 
$$
then $Q_i>0$ if $\sigma_i>0$ and vice versa.  Then $A_{ij}\ge 0$(i.e. $A_{ij}$ is a connectivity matrix for a directed network) 
implies that the non trivial equilibrium is $\sigma_i=1$: if $\sigma_k=0$ for some $k\in K$ then we have
$$
\sum_{j\notin K} A_{kj}\sigma_j=0
$$
so $A_{kj}=0$ for all $j\notin K$ and the network is disconnected. Then we have the trivial solution $\sigma_i=0$. For each equilibrium solution $\sigma^\ast$ we have a stability basin
$$
S_{\sigma^\ast}=\left \{\sigma \; | \; \lim_{t\to\infty} \sigma(t)=\sigma^\ast\right \}
$$
If $S_{\sigma^\ast}$ defined neighborhood of $\sigma^\ast$ the solution is stable or if $S_{\sigma^\ast}=\{\sigma^\ast\}$ the solution completely unstable. 
The stability of the origin depends on the existence of a Ljapounov function: let introduce the network activity
$$
\Sigma(t)=\sum_i \sigma_i(t)=\sum_i \Theta(Q_i(t-1))\ge \Sigma(t-1)
$$
since if each node has at least one input link, $A_{ij}=1$ implies $\sigma_j(t-1)\Rightarrow \sigma_i(t)=1$ and the activity cannot decrease. The solution $\sigma_i=0$
is completely unstable. If there would exists an equilibrium solution with $\sigma_k=0$ for some $k$ then we define $S_A$ the set of nodes s.t.
$$
i\in S_A\quad \Rightarrow \quad \sigma_i=0
$$
(obviously $\sigma_k\in S_A$). Let $S_{\bar A}$ the complement of $S_A$, the network dynamics implies
$$
0=\sum_j A_{ij}\sigma_j=\sum_{j\notin S_A} A_{ij}\sigma_j=0 \qquad \textrm{if}\quad i\in S_A
$$
so that $A_{ij}=0$ if $i\in S_A$ and $j\in S_{\bar A}$: i.e. there is not a cause-effect connection between $S_{\bar A}$ and $S_A$ and the state $\sigma_i=1$ for $i\in S_{\bar A}$
is an equilibrium state. Therefore we have as many equilibrium states as many partitions $S_A$ and $S_{\bar A}$ there exist such that $S_A$ triggers the activity of $S_{\bar A}$
but not vice versa. For any initial condition $\sigma_i(0)=\delta_{ik}$ the possible evolution are a periodic orbit or an equilibrium state: one can detect all the equilibrium conditions
by $\sigma^\ast$ by the condition
$$
\sigma_i^\ast=1\quad \textrm{if}\quad \sigma_i(t)=1\quad \textrm{for}\;\textrm{some}\quad t\ge 0
$$
The equilibrium states are a semigroup: let $\sigma^a$ and $\sigma^b$ two equilibrium states the
$$
\sigma^a\cup \sigma^b=\sigma^c
$$
is still an equilibrium. An example: if there exit a one directional loop $\gamma$ in the network and there is no output link
from $\gamma$ to the remaining nodes of the network then $\sigma_i=1$ for any $i\in \gamma$ is an equilibrium.
If the loop is simple (each node has a one input link and one output link) the equilibrium is neutral since any change
$\sigma_i=1\to\sigma_i=0$ creates a periodic orbit (the total activity is constant). But if we we add a link to the loop
then we get a stable solution since a single node can trigger the activity of two nodes and the equilibrium is an attractive
stationary state (see figure \label{fig:onecluster}). If a node is accidentally set to zero this anomaly propagates in the loop, 
until it reaches the node $4$ where it is annihilated by the activity of the node $(2)$. The average lifetime of a single perturbation
is the average path length to propagate to the node $(4)$ from the initial node (therefore it depends from the loop length or in case
of presence of many loops, the average path length is computed considering independent loops). 
\vskip .5 truecm 
\begin{figure}
\begin{center}
\begin{tikzpicture}
[->,>=stealth',shorten >=1pt,auto,node distance=3cm,
                    thick,main node/.style={circle,draw,font=\Large}]

  \node[main node] (1) {1};
  \node[main node] (2) [below left of=1] {2};
  \node[main node] (3) [below right of=2] {3};
  \node[main node] (4) [below right of=1] {4};

  \path[every node/.style={font=\small}]
    (1) edge node [left] {+1} (4)
    (2) edge node [right] {+1} (1)
        edge node {+1} (4)   
    (3) edge node [right] {+1} (2)
    (4) edge node [left] {+1} (3);
    \label{schema1}
\end{tikzpicture}
\end{center}
\caption{\emph{Example of random boolean network.}}
\label{fig:onecluster}
\end{figure}
\vskip .5 truecm
The boolean network models the propagation of information. By studying the stability problem of the solution $\sigma_i=1$ it is convenient to introduce the dual dynamics:
$$
\sigma_i^c(t+1)=\Theta\left (\prod_{j\sim i} A_{ij}\sigma^c_j(t)\right )=\prod_{j\sim i} A_{ij}\sigma^c_j(t)
$$
where $\sigma_i^c=1-\sigma_i$ is the dual state of the node and the product is restricted to the nodes connected to $i$
($A_{ij}\ne 0$): i.e. the node $(4)$ takes the state $\sigma^c=1$ only if both the nodes
$(1)$ and $(2)$ in that state at previous time. This dynamics is valid for any configuration of the network and the state $\sigma^c=1$
moves on the network until it reaches an absorbing state for which
$$
\prod_{j\sim i} \sigma^c_j(t)=0 \quad \forall \; i
$$
For a given stable equilibrium $\sigma^\gamma$ state associated to a loop $\gamma$ any environmental perturbation
that set to zero a activity of a node will destroy the equilibrium after a time equal to the number of the loop nodes minus one.
For example in the figure there are two loops $((1)\to (2)\to (3)\to (4))$ and $((2)\to(4)\to (3))$ if we set to zero the node 
$(4)$ after three iterations all the nodes will be in the zero state. The two loops are nit independent since one loops contains the other). On the contrary if we set to zero the node $(1)$ one loop remains active. This remark allows to introduce the concept of control node: a node is a control node if its state is able to 
force the state of the whole network. The effect of a thermal bath could be introduced by assuming that the state of a node
is defined as random variable that takes value $\sigma_i(t)=1$ with probability $p_i(t)$ where 
\begin{equation}
p_i(t+1)=\Theta_T\left (\sum_j A_{ij}\sigma_j(t)\right )
\label{stocdyn}
\end{equation}
and $\Theta_T(x)$ is a logistic function 
$$
\Theta_T(x)=\frac{1}{2}\left (1+\textrm{tgh}(x/T-\epsilon)\right )
$$
where $\epsilon$ measures to tendency of the network to be in the idle state when no stimulus is present.
The logistic function is a generic sigmoidal function we do not expect that the specific form of $\Theta_T(x)$ is critical for the results.
\par\noindent



Remark: since the values of $x$ are
quantized to integer in any case, if $\epsilon>T^{-1}$ the idle state is statistically attractive so $\epsilon$ could define a critical temperature
for the network activation. We recover the deterministic dynamics for $T\to 0$. As a stochastic process we have a Markov process (since
the realization of the variable $\sigma_i(t+1)$ depends only on the present state $\sigma_j(t)$ of the network. The dynamics (\ref{stocdyn})
is a Markov field: the realization of the variable $\sigma_i$ depends only from the present state of the network (and not from past states) and
only from the states of the connected nodes $A_{ij}\ne 0$. The last condition (Markov field) means that the realizations of $\sigma_i(t)$
and $\sigma_j(t-1)$ are independent if the nodes are not connected. The transition probabilities depend from the state of the network
and one derives the average dynamics
\begin{equation}
<\sigma_i>(t+1)=p_i(t+1)=\left \langle \Theta_T\left (\sum_j A_{ij}\sigma_j(t)\right ) \right \rangle\simeq
 \Theta_T\left (\sum_j A_{ij}p_j(t)\right ) 
\label{avestoc}
\end{equation}
Then we have two possibilities: if the total average network activity tends to increase
\begin{equation}
\bar \Sigma(t+1)=\sum_i p_i(t+1) =\sum_i  \Theta_T\left (\sum_j A_{ij}p_j(t)\right ) > \bar \Sigma(t)
\label{condave}
\end{equation}
the equilibrium solution $\sigma_i=1$ is attractive, on the contrary we have a an average tendency to decrease the network activity. 
The situation is illustrated in the fig. \label{fig:crit}
\begin{figure}[h]
\label{fig:crit} 
\centering{\includegraphics[scale=1.3]{images/theta.pdf}}
\caption{\emph{Possible behavior for the condition (\ref{condave}); the units are arbitrary and scale with the network dimension.}}
\end{figure}
Remark: the mean field approximation apply when the $\Theta_T(x)$ can be approximated by a linear function locally: i.e. the fluctuations are small
enough to approximate the function by a linear function in the whole fluctuation range. This is certainly not true when we have fat tail fluctuations.
\par\noindent
Except for a small initial region, the condition (\ref{condave}) can be satisfied up to a critical value of the network activity $\Sigma$ (if the temperature
is not too big), so that the average activity tends to increase. But if the activity is below the critical value then the network activity tend to decrease and
the stability of the solution $\sigma_i=1$ is lost. A connected network tends to be more stable since the quantities
$\sum_j A_{ij}\sigma_j$ increase.  This picture is clearly an approximation since we neglect the fluctuation effects: if the fluctuations are big (this
depends also on the connectivity matrix) we may have a fast transition between the two possible regime and a correction of the critical value. The
critical vale is a consequence of the sigmoidal behavior of the $\Theta_T(x)$ function and its depends on the temperature and on the $\epsilon$
values. In presence of fluctuations and of two dynamical regimes (active and non active) we expect that the network activity may switch from one regime
to another with a characteristic time scale (cfr. Kramer transition rate Theory). The transition may be triggered by large fluctuations that are both consequence
of rare events (in such a case the probability should be exponentially small with respect the activity) but also depend on the network structure (the presence of
hub nodes that can change the activity of many nodes amplifies the effect of small fluctuations (i.e. the change of the hub node state) and may introduce
fat tail statistic in the fluctuation distribution). A second stochastic effect is related to the fluctuations of the connectivity due to environmental causes: 
the matrix $A_ij(t)$ is a stochastic process (so that its entries change their value according to a probability distribution). The simplest model can be formulated as
follows: we assume that the nonzero entries $A_{ij}(t)$ assume value $1$ with a given probability $p$ (independent from the network state)
 each time step $\Delta t$ (i.e. we are not simulating
a parametric white noise, but a correlated random noise with a define correlation time scale $\Delta t$). $\Delta t$ is the shortest evolution time scale
for the system (we need a physical interpretation) and we set $\Delta t=1$. The effect of a parametric noise is substantially different from the environmental noise
and the evolution equation (\ref{evolnet}) reads
\begin{equation}
\sigma_i(t+1)=\Theta\left (\sum_j A_{ij}(t)\sigma_j(t)\right )
\label{evolnetstoc}
\end{equation}
In such a case the average dynamics is not useful and the problem can be studied by the representative dynamics
$$
\zeta_i(t+1)=\sum_j A_{ij}(t)\zeta_j(t)\qquad \Rightarrow \qquad \zeta(t)=\prod_{k=1}^t A(k)\zeta(0)
$$
where the solution is the product of random matrices (there are results on the spectral properties). From a biological point of view means
that the interaction of genes depends also by external factors. \par\noindent  
One could say that the network is active if a certain condition is satisfied (for example the average
total activity should overcome a given threshold) so that fluctuation may introduce the existence of non active states.
Problems for a single network: starting from a loops with a fixed dimension adds randomly links to stabilize the
equilibrium solutions (existence of sub-loops)  and study the robustness of the solution and the recovery times in relation with
the connectivity matrix; adding the temperature, study the existence of critical value for the appearance of the equilibrium solution; the thermodynamic limit. The effect of the environmental noise has to be justified from a biological point of view by relating it
to the individual variability of the cell phenotypes in an homogeneous population.\par\noindent
In the stochastic models one should also consider the problem that the connectivity matrix is not fixed (for example we have
a ensemble of admissible matrices or the existence of the links is a random event). In such a case we have a stochastic
dynamics
\begin{equation}
\sigma_i(t+1)=\Theta\left (\sum_j A_{ij}(t)\sigma_j(t)\right )
\label{stocdyn2}
\end{equation}
where $A_{ij}(t)$ is a random process with value $\in\{0,1\}$ maintaining some average properties of the connectivity;
this is an alternative to the environmental noise (parametric noise). This model simulates the fact that the activation of a link
may depends on random events (i.e. not only from the existence of the link) so that a genetic network is indeed a stochastic
network. In principle any realization of the connectivity matrix $A_{ij}$ has an equilibrium $\sigma_i=1$ but the robustness
of equilibrium can be influenced by the fluctuations. Problem: if the robustness of the equilibrium with respect to the external
perturbations (i.e. an external signal on a node) depends on the spectral properties of the connectivity matrix then
it is possible to study the spectral properties of random connectivity matrices and develop a control theory
for the network. 
\par\noindent
Let us consider the existence of competitive networks (see Figure \ref{fig:comp}) that are linked by inhibitory links: if the first network is
in an exited state the second network should be completely switched off for a stable equilibrium.
\vskip .5 truecm

\begin{figure}
\centering
\begin{tikzpicture}[->,>=stealth',shorten >=1pt,auto,node distance=2.5cm,
                    thick,main node/.style={circle,draw,font=\Large}]

  \node[main node] (1) {1};
  \node[main node] (2) [below left of=1] {2};
  \node[main node] (3) [below right of=2] {3};
  \node[main node] (4) [below right of=1] {4};
  \node[main node] (5) [right of=4] {5}; 
  \node[main node] (6) [below right of=5] {6};
  \node[main node] (7) [above right of=6] {7};
  \node[main node] (8) [above left of=7] {8};

\path[every node/.style={font=\small}]
    (1) edge node [left] {+1} (4)
         edge node {-1} (8)
    (2) edge node [right] {+1} (1)
         edge node {+1} (4)   
    (3) edge node [right] {+1} (2)
    (4) edge node [left] {+1} (3)
    (8) edge node [left] {+1} (7)
    (5) edge node [right] {+1} (8) 
    (6) edge node [right] {+1} (5)
        edge node {-1} (3)
    (7) edge node [left] {+1} (6)
            edge node {+1} (5)  ;
\end{tikzpicture}
\caption{\emph{Example of a network composed by two competitive subnetworks.}}
\label{fig:comp}
\end{figure}
\par\noindent
If we start with all the node states set to one we create a frustrated situation, otherwise the network choose one
of the two possible stable states. In such a case the presence of an environmental noise could induce the transition
to one state to another (to be studied). An external forcing breaks the symmetry.\par\noindent
We expect a transition phase as a function of the temperature: increasing the temperature the node states tends 
to be independent, but under a threshold the system should choose a stationary state. 
\par\noindent
The system can be generalized to consider the interactions of different cooperative networks (possibly with different internal
structure) that are connected by inhibitory links (in the case of connection with excitatory links we join the subnetwork in a 
single one). We can introduce a metadynamics where $\nu_k(t)$ is the state of the $k$ subnetwork and
we have a relation
\begin{equation}
\nu_k(t+\Delta t)-\nu_k(t)=\phi(\nu_k(t))-\gamma\left (H_{kj}\nu_j(t)\right )
\label{metastoc}
\end{equation}
where $H_{hk}\ge 0$ is an inhibitory connectivity matrix. $\phi(\nu_k(t))$ describes the tendency of the sub-network to increase
its activity and $\gamma$ the average decreasing of the activity due to the presence of other sub-networks.
This is an effective equation: $\nu_k$ should describe the network activity (i.e. it could be the time-average activity of the nodes
assuming that the network could be considered in a stationary state). Indeed the evolution time scale $\Delta t$ could be assumed
$\Delta t\gg 1$ so that the subnetwork states are relaxed to a stationary states. 
The structure of attraction basins of the stable states could be related to a potential in the state space if
$$
\nu_k(t+1)-\nu_k(t)=-\frac{\partial }{\partial \nu_k}\left [ \frac{\gamma}{2} \sum_{ij}\nu_i H_{ij}\nu_j+\sum_j V(\nu_j)\right ]
$$
where 
$$
\phi(\nu)=-\frac{\partial V}{\partial \nu}
$$
Since $\phi(\nu)\ge 0$ $V(\nu)$ is increasing. Then we introduce the energy
$$
E=\frac{\gamma}{2} \sum_{ij}\nu_i H_{ij}\nu_j+\sum_j V(\nu_j)
$$
and the equilibrium are the critical points of the energy. Moreover 
\begin{eqnarray}
E(t+1)-E(t)&\simeq& (\nu_k(t+1)-\nu_k(t))\frac{\partial}{\partial \nu_k}\left [\frac{\gamma}{2} \sum_{ij}\nu_i(t) H_{ij}\nu_j(t)+\sum_j V(\nu_j(t))\right ]\nonumber \\
&=& -\frac{1}{2}\frac{\partial}{\partial \nu_k}\left [ \frac{\gamma}{2} \sum_{ij}\nu_i(t) H_{ij}\nu_j(t)+\sum_j V(\nu_j(t)) \right ]^2\nonumber
\end{eqnarray}
Therefore the energy is a Ljapounov function and the system equilibria are defined by the critical points of the Energy function
corresponding to local minima and maxima.\par\noindent
Remark: the existence of the Energy implies that $H_{ij}$ is symmetric negative defined
$$
\frac{\partial^2 E}{\partial \nu_j \partial \nu_i}=\frac{\partial^2 E}{\partial \nu_i \partial \nu_j}
$$
\par\noindent
The stochastic effect has to be introduce but it is possible a thermodynamics approach and a thermodynamics equilibrium exists
according to the Maxwell-Boltzmann distribution and the detailed balance condition. This means that the whole network does not
satisfy this condition, but the metadynamic network realized a reversible Markov process. The existence of a thermodynamic equilibrium
allows to use Maximal Entropy Principle and the Maxwell Boltzmann distribution when we introduce a thermal bath. 
\par\noindent
We are interested in networks with many different equilibria each one related to exited state of subnetworks (or a combination of subnetworks), 
in the effect of a thermal noise and in the effect of external forcing. The external  we introduce in the network boundary nodes whose state is defined by a given external signal
$\sigma_b(t)$ (possible a stochastic process) then the network dynamics reads
$$
\sigma_i(t+1)=\Theta\left (\sum_j A_{ij}\sigma_j(t)+\sum_b A_{ib}\sigma_b(t)\right )
$$
where $A_{ib}$ is the link between the environmental node $b$ and the node $i$. It is possible to introduce a probabilistic description of the evolution
of the probability that the network is in the state $\sigma'$ at time $t+1$ according to
\begin{equation}
p(\sigma',t+1)=\sum_{\sigma}\pi(\sigma',\sigma)p(\sigma,t)
\label{lapla}
\end{equation}
where
$$
\pi(\sigma'|\sigma)=E\left (\delta_{\sigma',\Phi_{\sigma_b}(\sigma)} \right )
$$
and the expectation value is computed on the realization of the input noise. $\pi(\sigma'|\sigma)$ is the transition rate per unit time (the continuous limit could be considered).
Let $\sigma_{eq}$ stable equilibrium state the effect of external random perturbations could be to move the network state in a neighborhood of the equilibrium solution
or it could induce a transition to other equilibrium basin attractions so that the dynamics starts to perform an intermittence behavior. In such a case the relevant quantities are  
the residence times in the different basins that can be associated to metastable states.
\par\noindent
We introduce the stochasticity in the system assuming that the adjacency matrix is not known: i.e. $A_{ij}$ is a extracted from an ensemble of random matrices.
As the result of an experimental one could assume that each entry $A_{ij}$ is a dichotomous random variable with probability $p_{ij}$ to get the value $\pm 1$ (i.e. the
link is active). The value $p_{ij}=0$ is admitted so that the corresponding link it always inactive. The problems are:
\begin{enumerate}
    	\item Classifying the equilibrium states in relation to their robustness with respect the changes in the adjacency matrix;
     \item Understanding the representativity of the average dynamics: i.e. substituting the adjacency matrix with an average matrix one highlights the dynamical properties 
     that are correctly described by the average system
     \item Pointing out the existence of bifurcation phenomena so that it is possible to divide the ensemble in different communities with similar dynamical behaviors.
\end{enumerate}


\chapter{Analysis}\label{analysis}
\lhead[\fancyplain{}{\bfseries\thepage}]{\fancyplain{}{\bfseries\rightmark}}

In this Chapter we explain the starting implementation and analysis of the model following biological considerations.
\section{Analysis of the model}

\chapter*{Conclusions}
\lhead[\fancyplain{}{\bfseries\thepage}]{\fancyplain{}{\bfseries\rightmark}}

In this work we proposed a theoretical model for cell differentiation.
Since this biological process is governed by Gene Regulatory Networks, these networks can be modelled by Random Boolean Networks, in which each gene can be represented by  node which can be "on" or "off".
The process of differentiation is a multistable dynamical system, and involves different type of cells, but all of these start from one unique type of cell: the stem cells. The complexity of this process lays in the fact that cells "can" decide if transforming in one type of cell with respect one other, and for this reason seems pheasible that if a network for a type of cell is active, it may hinibit the network for a different type of cell.
The proposed model can give an estimate of a fitness potential for two different type of cells, according to the theory of Waddington.
This process concerns also the birth of cancer cells: cancers cell can be governed by a specific type of regulatory network, which often is inactive, but due to some external stimuli it can be activated and gives an irreversible process of production of cancer cells. This can be seen as a local minima of the Waddington potential, which is impossible to escape.
The role of noise in this model is crucial, but it is well known that biological process are indeed very dependent to external noise.

Today, constructing Gene Regulatory Networks from sequencing data is still impossible, so future studies on this model can give more details about multiple type of cells.

\begin{thebibliography}{90}             %crea l'ambiente bibliografia
\rhead[\fancyplain{}{\bfseries \leftmark}]{\fancyplain{}{\bfseries
\thepage}}
%%%%%%%%%%%%%%%%%%%%%%%%%%%%%%%%%%%%%%%%%aggiunge la voce Bibliografia
                                        %   nell'indice
\addcontentsline{toc}{chapter}{Bibliography}
%%%%%%%%%%%%%%%%%%%%%%%%%%%%%%%%%%%%%%%%%provare anche questo comando:
%%%%%%%%%%%\addcontentsline{toc}{chapter}{\numberline{}{Bibliografia}}

\bibitem{K39} Janeway, \emph{Immunobiology}, 9th Edition
\bibitem{K40} E. Davidson, M. Levine, \emph{Gene Regulatory Networks}, doi:10.1073/pnas.0502024102, (2005)
\bibitem{K41} A. Wuensche, \emph{Genomic regulation modeled as a network with basins of attraction}, Pacific Symposium on Biocomputing. Pacific Symposium on Biocomputing, (1998)

\bibitem{K49} S. A. Kauffman, \emph{Investigations}, Oxford University
Press, (2000)
\bibitem{K1} S. A. Kauffman, \emph{Metabolic Stability and Epigenesis in
Randomly Constructed Genetic Nets}, J. Theoret. Biol. (1969)
\bibitem{K42} MacArthur S. et al., \emph{Developmental roles of 21 Drosophila transcription factors are determined by quantitative differences in binding to an overlapping set of thousands of genomic regions},doi:10.1186/gb-2009-10-7-r80 (2009)
\bibitem{K7} S. A. Kauffman: J. Theor. Biol., 44, Physica D, 10, 145 (1984)
\bibitem{K43} Peccoud, J. and Ycart, B., \emph{Markovian Modelling of Gene Products Synthesis.}, https://doi.org/10.1006/tpbi.1995.1027, (1995)
\bibitem{K44} T.B. Kepler and T.C. Elston, \emph{Stochasticity in transcriptional regulation: origins, consequences, and mathematical representations}, 10.1016/S0006-3495(01)75949-8 , (2001)
\bibitem{K45} J.M. Pedraza, J. Paulsson, \emph{Effects of molecular memory and bursting on fluctuations in gene expression}, 10.1126/science.1144331, (2008).
\bibitem{K46} Y. Sasai \emph{Cytosystems dynamics in self-organization of tissue architecture}, https://doi.org/10.1038/nature11859, (2013)

\bibitem{K47} B. Zhang and P. G. Wolynes, \emph{Stem cell differentiation as a many-body problem}, https://doi.org/10.1073/pnas.1408561111, (2014)
\bibitem{K48} Zhou et al., \emph{Fast Pyrolysis of Glucose-Based Carbohydrates withAdded NaCl Part 2: Validation and Evaluation of theMechanistic Model},DOI 10.1002/aic.15107, (2016)
\bibitem{K8} B. Derrida, \emph{Random Networks of Automata: A Simple Annealed
Approximat ion.}, (1985)
\bibitem{K5} Drossel B.,\emph{Random Boolean Networks},arXiv:0706.3351 ,(2008)

\bibitem{K3} R.Serra, M. Villani, A. Barbieri, S.A. Kauffman, A. Colacci,\emph{On the dynamics of random Boolean networks subject to noise:
Attractors, ergodic sets and cell types.},J Theor Biol 265: 185–193, (2010)
\bibitem{K2} M. Villani, A. Barbieri, R. Serra,\emph{A Dynamical Model of Genetic Networks for Cell Differentiation}, doi:10.1371/journal.pone.0017703.g001,(2011)

\bibitem{K4} S. Huang, I. Ernberg, S. Kauffman,\emph{Cancer attractors: A systems view of tumors from a gene network
dynamics and developmental perspective}, doi:10.1016/j.semcdb.2009.07.003, (2009)
\bibitem{K6} S. Kauffman, \emph{A proposal for using the ensemble approach to understand
genetic regulatory networks},Journal of Theoretical Biology 230 (2004) 581–590 ,(2004)
\bibitem{K9} M. Ali Al-Radhawi , Nithin S. Kumar, Eduardo D. Sontag , Domitilla Del Vecchio, \emph{Stochastic multistationarity in a model of the hematopoietic
stem cell differentiation network},doi:10.1109/cdc.2018.8619300, (2018)
\bibitem{K10} Cameron P. Gallivan, Honglei Ren and Elizabeth L. Read,\emph{Analysis of Single-Cell Gene Pair
Coexpression Landscapes by
Stochastic Kinetic Modeling Reveals
Gene-Pair Interactions in
Development},doi: 10.3389/fgene.2019.01387 ,(2019)

\bibitem{K11} Jifan Shi, Tiejun Li , Luonan Chen, Kazuyuki Aihara,\emph{Quantifying pluripotency landscape of cell
differentiation from scRNA-seq data by
continuous birth-death process},https://doi.org/10.1371/journal.
pcbi.1007488 ,(2019)
\bibitem{K12} Jin Wang, Kun Zhang, Li Xu, and Erkang Wang ,\emph{Quantifying the Waddington landscape and biological
paths for development and differentiation}, https://doi.org/10.1073/pnas.1017017108 ,(2011)
\bibitem{K13} Waddington CH, \emph{The strategy of the genes: a discussion of some aspects of
theoretical biology}. London: Allen and Unwin, (1957)


\bibitem{K14} B. Drossel,\emph{Random Boolean Networks}, arXiv:0706.3351. (2008)
\bibitem{K15} M. Rybarsch and S. Bornholdt,\emph{On the dangers of Boolean networks:
Activity dependent criticality and threshold networks not faithful to biology}, arXiv:1012.3287v1. (2010)
\bibitem{K16} J. Park and M. E. J. Newman,\emph{The statistical mechanics of networks}, DOI: 10.1103/PhysRevE.70.066117 (2004)
\bibitem{K17} C. Gershenso,\emph{Introduction to Random Boolean Networks}, arXiv:nlin/0408006,(2004)
\bibitem{K17} B. Derrida and H.Flyvbjerg, \emph{The random map model: a disordere model with deterministic dynamics}, J.Physique, (1987)
\bibitem{K19} R. V. Solè, B. Loque, \emph{Phase transitions and antichaos in generalized Kauffman networks},Physics Letters,(1994)
\bibitem{K20} J. T. Lizier, S. Pritam, M. Prokopenko, \emph{Information dynamics in small-world Boolean networks},, (2011)
\bibitem{K21} B. Derrida, \emph{Spin glasses, random boolean networks and simple models of evolution}
\bibitem{K22} A. Rèka and A-L. Barabàsi, \emph{Statistical mechanics of complex networks}, Reviewes of modern physics, Volume 74,(2002)
\bibitem{K23} N. Masuda , M. A. Porter, R. Lambiotte ,\emph{Random walks and diffusion on networks},Physics Reports 716–717 1–58,(2017)
\bibitem{K24} T. Biyikoglu, J. Leydold, P. F. Stadler,\emph{Laplacian Eigenvectors of Graphs}, Springer
\bibitem{K25} Fan R. K. Chung,\emph{Spectral Graph Theory}, CBMS
\bibitem{K26} Sui Huang , Ingemar Ernberg, and Stuart Kauffman,\emph{Cancer attractors: A systems view of tumors from a gene network
dynamics and developmental perspective}, DOI:10.1016/j.semcdb.2009.07.003, (2009)
\bibitem{K27} M. Ali Al-Radhawi, Nithin S. Kumar, Eduardo D. Sontag, Domitilla Del Vecchio ,\emph{Stochastic multistationarity in a model of the hematopoietic
stem cell differentiation network},DOI:10.1109/cdc.2018.8619300.,(2018)
\bibitem{K28} Rushina Shah, Domitilla Del Vecchio,\emph{Reprogramming cooperative monotone dynamical systems},DOI:10.1109/cdc.2018.8618649, (2018)
\bibitem{K29} Atefeh Taherian Fard and Mark A. Ragan, \emph{Modeling the Attractor Landscape of
Disease Progression: a
Network-Based Approach},DOI: 10.3389/fgene.2017.00048, (2017)
\bibitem{K30} Sui Huang, Yan-Ping Guo, Gillian May, Tariq Enver,\emph{Bifurcation dynamics in lineage-commitment in bipotent progenitor cells},Developmental Biology 305, (2007)
\bibitem{K31} Xin-She Yang and Young Z. L. Yang,\emph{Cellular Automata Networks},arXiv:1003.4958 , (2010)
\bibitem{K32} Christopher H. Joyner and Uzy Smilansky, \emph{Dyson’s Brownian-motion model for random matrix
theory - revisited},arXiv:1503.06417,(2015)
\bibitem{K33} Cameron P. Gallivan, Honglei Ren and Elizabeth L. Read,\emph{Analysis of Single-Cell Gene Pair
Coexpression Landscapes by
Stochastic Kinetic Modeling Reveals
Gene-Pair Interactions in
Development} ,doi: 10.3389/fgene.2019.01387,(2019)

\bibitem{K34} Xin Kang, Chunhe Li,\emph{Landscape inferred from gene expression data governs pluripotency in
embryonic stem cells},omputational and Structural Biotechnology Journal 18 (2020) 366–374,(2020)
\bibitem{K35} Genaro J. Martı́nez , Andrew Adamatzky  ,
Bo Chen , Fangyue Chen , Juan C.S.T. Mora ,\emph{Simple networks on complex cellular automata:
From de Bruijn diagrams to jump-graphs},(2017)

\bibitem{K37} Chen L et al., \emph{Biomolecular networks: methods and applications in systems biology}, Wiley, Hoboken ,(2009)

\bibitem{K38} Schnakenberg J., \emph{Network theory of microscopic and macroscopic behavior of master equation systems.}, Reviews of Modern Physics,Vol.48, (1976)
\bibitem{K50} C.W. Gardiner, \emph{Handbook of Stochastic Methods}, Springer, (1985)



\end{thebibliography}
%%%%%%%%%%%%%%%%%%%%%%%%%%%%%%%%%%%%%%%%%non numera l'ultima pagina sinistra
\clearpage{\pagestyle{empty}\cleardoublepage}


\end{document}


\chapter*{Introduction}                 %crea l'introduzione (un capitolo
                                        %   non numerato)
%%%%%%%%%%%%%%%%%%%%%%%%%%%%%%%%%%%%%%%%%imposta l'intestazione di pagina
\rhead[\fancyplain{}{\bfseries
INTRODUCTION}]{\fancyplain{}{\bfseries\thepage}}
\lhead[\fancyplain{}{\bfseries\thepage}]{\fancyplain{}{\bfseries
INTRODUCTION}}
%%%%%%%%%%%%%%%%%%%%%%%%%%%%%%%%%%%%%%%%%aggiunge la voce Introduzione
                                        %   nell'indice
\addcontentsline{toc}{chapter}{Introduction}

The cell is a paradigmatic example of complex system governed by many processes which are not yet understood: the process of cell differentiation is one of these. 
Cell differentiation is the dynamical process in which stem cells reproduce and give arise to different type of cells.
Waddington in 1957 \cite{K13} proposed to model cell differentiation with an epigenetic landscape in which lay different type of cells. 
This epigenetic landscape can be seen as a potential in a physical system in which different type of cells are attracted by the different wells of this potential.
From this model, recent studies on omics data propose ways to find epigenetic landscape for different cells\cite{K10}\cite{K11}\cite{K12}, using stochastic processes.

Now, it is well known that cell differentiation is governed by the so called Gene Regulatory Networks (GRNs).
A GRN is a collection of molecular regulators that interact with each others and with other substances in the cell to define the gene expression levels of mRNA and proteins. 

Disruption of these processes by inappropriate regulatory signals and by mutational rewiring of the
network can lead to tumorigenesis\cite{K4}. 

Kauffman proposed for the first time in 1969\cite{K1} to model GRNs through the so called Random Boolean Networks (RBN).
RBN are networks in which each node can have only two possible values: 0 or 1, where each node represents a gene in GRN which can be "on" or "off"\cite{K5}.
The evolution of the state of the network is given by some boolean functions, depending on the connectivity of the nodes. So each node will have one boolean function which defines the next state during the descrete evolution.
From this networks, one can find some periodical structures called \emph{attractors}, which can be associated to different type of cells by a biological point of view\cite{K2}\cite{K3}\cite{K9}.

Omics data available are subject to noise, and we can’t expect to see directly connections between different genes, so we want to build a model able to see some statistical properties among these networks.


\chapter{Cell Differentiation and cancer cells}\label{celldiff}
\lhead[\fancyplain{}{\bfseries\thepage}]{\fancyplain{}{\bfseries\rightmark}}


\section{Gene Regulatory Networks}

All steps of gene expression can be modulated, since passage of the transcription of DNA to RNA, to the post-translational modification of the protein
produced. Hence, gene expression is a complex process regulated at several stages in the synthesis of proteins. In addition to the DNA transcription regulation, the expression of a gene may be controlled during RNA processing and transport (in eukaryotes), RNA translation, and the post-translational modification of proteins. This gives rise to genetic regulatory systems structured by networks of regulatory interactions between DNA, RNA, proteins and other molecules [6]: a complex network termed as a gene regulatory
network (GRN). Some, noteworthy, kind of proteins are the transcription factors that bind to specific DNA sequences in order to regulate the
expression of a given gene. The power of transcription factors resides in their ability to activate and/or repress transcription of genes. The activation of
a gene is also referred to positive regulation, while the negative regulation
identifies the inhibition of the gene.
The regulation of gene expression is essential for the cell, because it
allows to control the internal and external functions of the cell. Furthermore,
in multicellular organisms, gene regulation drives the processes of cellular
differentiation and morphogenesis, leading to the creation of different cell
types that possess different gene expression profiles, and these last therefore
produce different proteins that have different ultrastructures that suit them
to their functions (though they all possess the genotype, which follows the
same genome sequence) 4 . Therefore, with few exceptions, all cells in an
organism contain the same genetic material [6], and hence the same genome
(the haploid set of chromosomes of a cell). The difference between the cells
are emergent and due to regulatory mechanisms which can turn on or off
genes. Two cells are different, if they have different subsets of active genes.

\section{Cell Differentiation}
Cell differentiation is the process whereby stem cells become progressively
more specialized. The differentiation process occurs both during the devel-
opment of a multicellular organism and during tissue repair and cell turnover
in the adulthood. Gene expression, and therefore its regulatory mechanisms,
plays a critical role in cell differentiation; as described in the previous section.
Stem cells are undifferentiated biological cells which can both reproduce
themselves, self-renewal ability, and differentiate into specialized cells, po-
tency.

\chapter{The model}
\lhead[\fancyplain{}{\bfseries\thepage}]{\fancyplain{}{\bfseries\rightmark}}

\section{Stochastic models for dynamical systems on graphs}
%
We consider a physical system that can be described by an weighted interaction network among nodes that can assume different
dynamical states (in the case of a gene network the states $\sigma\in [0,1]$ and we have models similar to spin models).
The interaction structure is defined by signed adjacency matrix $A_{ij}\in[-1,0,1]$ where the sign refers to a cooperative or antagonist interaction between the connected nodes. 
In the simplest case, we introduce a stochastic dynamics using the probability $p_i(\sigma, t)$ that the node $i$ is in the state 
$\sigma$ (we assume $\sigma>0$) at time $t$: in a deterministic approach $p_i(\sigma, t)=\delta(\sigma-\sigma(t))$
to denote that the node assume the state $\sigma=1$ with probability one. The evolution of a deterministic model
can be described by the equation
$$
\sigma_i(t+1)=\Phi_i(\sigma(t))=\Theta\left (\sum_j A_{ij}\sigma_j(t)\right )
$$
where $\Theta(x)\in[0,1]$ is a threshold sigmoidal function (we assume $A_{ii}=0$ to avoid self loops). 
Remark: the dynamics is an information diffusion network. If we consider the linear system:
$$
\zeta_i(t+1)=\sum_jA_{ij}\zeta_j(t)
$$
where $\zeta_i$ are non negative integers we have an equivalent dynamics since $\sigma_i=\Theta(\zeta_i)$ and it is possible to study the linear system to derive some properties of the initial system. For example the relaxation time to the solution $\sigma_i = \ \forall i$ is for a given initial condition $\sigma_j(0)=\delta_{jk}$ is $t=n$ such that the matrix $A^n$ has positive entries along the whole $k$-th column. This means that for each node $i$ there is a walk of length $n$ from the initial node $k$ to $i$.

We also assume a cause-effect relation so that $A_{ij}$ is a directed
graph. The deterministic model is a Hopfield network (each node has at least an input and an output link; the environment nodes has only output links)
and one could study the equilibrium states and their stability. An equilibrium condition as follows is characterized as follows:
for each $i$ let
$$
Q_i(t)=\sum_j A_{ij}\sigma_j(t) 
$$
then $Q_i>0$ if $\sigma_i>0$ and vice versa.  Then $A_{ij}\ge 0$(i.e. $A_{ij}$ is a connectivity matrix for a directed network) 
implies that the non trivial equilibrium is $\sigma_i=1$: if $\sigma_k=0$ for some $k\in K$ then we have
$$
\sum_{j\notin K} A_{kj}\sigma_j=0
$$
so $A_{kj}=0$ for all $j\notin K$ and the network is disconnected. Then we have the trivial solution $\sigma_i=0$. For each equilibrium solution $\sigma^\ast$ we have a stability basin
$$
S_{\sigma^\ast}=\left \{\sigma \; | \; \lim_{t\to\infty} \sigma(t)=\sigma^\ast\right \}
$$
If $S_{\sigma^\ast}$ defined neighborhood of $\sigma^\ast$ the solution is stable or if $S_{\sigma^\ast}=\{\sigma^\ast\}$ the solution completely unstable. 
The stability of the origin depends on the existence of a Ljapounov function: let introduce the network activity
$$
\Sigma(t)=\sum_i \sigma_(t)=\sum_i \Theta(Q_i(t-1))\ge \Sigma(t-1)
$$
since if each node has at least one input link, $A_{ij}=1$ implies $\sigma_j(t-1)\Rightarrow \sigma_i(t)=1$ and the activity cannot decrease. The solution $\sigma_i=0$
is completely unstable. If there would exists an equilibrium solution with $\sigma_k=0$ for some $k$ then we define $S_A$ the set of nodes s.t.
$$
i\in S_A\quad \Rightarrow \quad \sigma_i=0
$$
(obviously $\sigma_k\in S_A$). Let $S_{\bar A}$ the complement of $S_A$, the network dynamics implies
$$
0=\sum_j A_{ij}\sigma_j=\sum_{j\notin S_A} A_{ij}\sigma_j=0 \qquad \textrm{if}\quad i\in S_A
$$
so that $A_{ij}=0$ if $i\in S_A$ and $j\in S_{\bar A}$: i.e. there is not a cause-effect connection between $S_{\bar A}$ and $S_A$ and the state $\sigma_i=1$ for $i\in S_{\bar A}$
is an equilibrium state. Therefore we have as many equilibrium states as many partitions $S_A$ and $S_{\bar A}$ there exist such that $S_A$ triggers the activity of $S_{\bar A}$
but not vice versa. For any initial condition $\sigma_i(0)=\delta_{ik}$ the possible evolution are a periodic orbit or an equilibrium state: one can detect all the equilibrium conditions
by $\sigma^\ast$ by the condition
$$
\sigma_i^\ast=1\quad \textrm{if}\quad \sigma_i(t)=1\quad \textrm{for}\;\textrm{some}\quad t\ge 0
$$
The equilibrium states are a semigroup: let $\sigma^a$ and $\sigma^b$ two equilibrium states the
$$
\sigma^a\cup \sigma^b=\sigma^c
$$
is still an equilibrium. An example: if there exit a one directional loop $\gamma$ in the network and there is no output link
from $\gamma$ to the remaining nodes of the network then $\sigma_i=1$ for any $i\in \gamma$ is an equilibrium.
If the loop is simple (each node has a one input link and one output link) the equilibrium is neutral since any change
$\sigma_i=1\to\sigma_i=0$ creates a periodic orbit (the total activity is constant). But if we we add a link to the loop
then we get a stable solution since a single node can trigger the activity of two nodes and the equilibrium is an attractive
stationary state (see figure).
\vskip .5 truecm
\begin{tikzpicture}[->,>=stealth',shorten >=1pt,auto,node distance=3cm,
                    thick,main node/.style={circle,draw,font=\sffamily\Large\bfseries}]

  \node[main node] (1) {1};
  \node[main node] (2) [below left of=1] {2};
  \node[main node] (3) [below right of=2] {3};
  \node[main node] (4) [below right of=1] {4};

  \path[every node/.style={font=\sffamily\small}]
    (1) edge node [left] {+1} (4)
    (2) edge node [right] {+1} (1)
        edge node {+1} (4)   
    (3) edge node [right] {+1} (2)
    (4) edge node [left] {+1} (3);
\end{tikzpicture}
\vskip .5 truecm

The boolean network models the propagation of information. By studying the stability problem of the solution $\sigma_i = 1$ it is convenient to introduce the dual dynamics:
$$
\sigma_i^c(t+1) = \Theta\bigl(\prod_{j~i}A_{ij}\sigma_j^c(t)\bigr) = \prod_{j~i}A_{ij}\sigma_j^c(t)
$$
where $\sigma_i^c=1-\sigma_i$ is the dual state of the node and the product is restricted to the nodes connected to $i$ $(A_{ij}\noteq 0 ) $: i.e. the node $(4)$ takes the state $\sigma^c=1$ only both the nodes (1) and (2) in that state at previous time. This dynamics is valid for any configuration of the network and the state $\sigma^c=1$ moves on the network until it reaches an absorbing state for which 
$$
\prod_{j~i} \sigma_j^c(t)=0 \forall i 
$$

For a given stable equilibrium $\sigma^\gamma$ state associated to a loop $\gamma$ any environmental perturbation
that set to zero a activity of a node will destroy the equilibrium after a time equal to the number of the loop nodes minus one.
For example in the figure there are two loops $((1)\to (2)\to (3)\to (4))$ and $((2)\to(4)\to (3))$ if we set to zero the node 
$(4)$ after three iterations all the nodes will be in the zero state. The two loops are not independent since one loops contains the other.On the contrary if we set to zero the node $(1)$ one loop remains active. This remark allows to introduce the concept of control node: a node is a control node if its state is able to 
force the state of the whole network. The effect of a thermal bath could be introduced by assuming that the state of a node
is defined as random variable that takes value $\sigma_i(t)=1$ with probability $p_i(t)$ where 
\begin{equation}
p_i(t+1)=\Theta_T\left (\sum_j A_{ij}\sigma_j(t)\right )
\label{stocdyn}
\end{equation}
and $\Theta_T(x)$ is a logistic function 
$$
\Theta_T(x)=\frac{1}{2}\left (1+\textrm{tgh}((x-\epsilon)/T)\right )
$$
where $eps$ measures the tendency of the network to be in the idle state when no stimulus is present. Remark: since the values of $x$ are quantized to integer in any case, if $eps>T^{-1}$ the idle state is statistically attractive so $eps$ could define a critical temperature for the network activation.
We recover the deterministic dynamics for $T\to 0$ and $\epsilon\to o(T)$, but the limit is singular, whereas $T\to\infty$.
As a stochastic process we have a Markov process (since the realization of the variable $\sigma_i(t+1)$ depends only on the present state $\sigma_j(t)$ of the network. The dynamics (2) is a Markov field: the realization of the variable $\sigma_i$ depends only from the present state of the network (and not from past states) and only from the states of the connected nodes $A_{ij}\noteq 0 $. The last condition (Markov field) means that the realizations of $\sigma_i(t)$ and $\sigma_j(t-1)$ are independent if the nodes are not connected. The transition probabilities depend from the state of the network and one derives the average dynamics
$$
<\sigma_i(t+1)> = p_i(t+1) = \< \Theta_T\bigl(\sum_jA_{ij}\sigma_j(t)\bigr)> = \Theta_T \bigl( \sum_jA_{ij}p_j(t)\bigr)
$$
Then we have two possibilities: if the total average network activity tends to increase
$$
\Sigma(t+1) = \sum_ip_i(t+1)=\sum_i\Theta_T\bigl(\sum_jA_{ij}p_j(t)\bigr)>\sum(t)
\label{eq:sigma}
$$
the equilibrium solution $\sigma_i = 1$ is attractive, on the contrary we have an average tendency to decrease the network activity. (FIGGG)
Remark: the mean field approximation apply when the $\Theta_T(x)$ can be approximated by a linear function locally: i.e. the fluctuations are small enough to approximate the function by a linear function in the whole fluctuation range. This is certainly not true when we have a fat tail fluctuations.
Except for a small initial region, the condition ~\eqref{eq:sigma} can be satisfied up to a critical value of the network activity $\Sigma$ ( if the temperature is not too big), so that the average activity tends to increase. But if the activity is below the critical value then the network activity tends to be more stable since the quantities $\sum_jA_{ij}\sigma_j$ increase.
we have a completely random behavior. We consider the case $T$ small (so $\epsilon$ small) to study the stability
of the equilibrium solution. One could say that the network is active if a certain condition is satisfied (for example the average
total activity should overcome a given threshold) so that fluctuation may introduce the existence of non active states.
Problems for a single network: starting from a loops with a fixed dimension adds randomly links to stabilize the
equilibrium solutions (existence of sub-loops)  and study the robustness of the solution and the recovery times in relation with
the connectivity matrix; adding the temperature, study the existence of critical value for the appearance of the equilibrium solution; the thermodynamic limit. The effect of the environmental noise has to be justified from a biological point of view by relating it
to the individual variability of the cell phenotypes in an homogeneous population.\par\noindent
In the stochastic models one should also consider the problem that the connectivity matrix is not fixed (for example we have
a ensemble of admissible matrices or the existence of the links is a random event). In such a case we have a stochastic
dynamics
\begin{equation}
\sigma_i(t+1)=\Theta\left (\sum_j A_{ij}(t)\sigma_j(t)\right )
\label{stocdyn2}
\end{equation}
where $A_{ij}(t)$ is a random process with value $\in\{0,1\}$ maintaining some average properties of the connectivity;
this is an alternative to the environmental noise (parametric noise). This model simulates the fact that the activation of a link
may depends on random events (i.e. not only from the existence of the link) so that a genetic network is indeed a stochastic
network. In principle any realization of the connectivity matrix $A_{ij}$ has an equilibrium $\sigma_i=1$ but the robustness
of equilibrium can be influenced by the fluctuations. Problem: if the robustness of the equilibrium with respect to the external
perturbations (i.e. an external signal on a node) depends on the spectral properties of the connectivity matrix (to be studied) then
it is possible to study the spectral properties of random connectivity matrices (see Catanzaro thesis) and develop a control theory
for the network. 
\par\noindent
Let us consider the existence of competitive networks (see figure) that are linked by inhibitory links: if the first network is
in an exited state the second network should be completely switched off for a stable equilibrium.
\vskip .5 truecm
\begin{tikzpicture}[->,>=stealth',shorten >=1pt,auto,node distance=2.5cm,
                    thick,main node/.style={circle,draw,font=\sffamily\Large\bfseries}]

  \node[main node] (1) {1};
  \node[main node] (2) [below left of=1] {2};
  \node[main node] (3) [below right of=2] {3};
  \node[main node] (4) [below right of=1] {4};
  \node[main node] (5) [right of=4] {5}; 
  \node[main node] (6) [below right of=5] {6};
  \node[main node] (7) [above right of=6] {7};
  \node[main node] (8) [above left of=7] {8};

\path[every node/.style={font=\sffamily\small}]
    (1) edge node [left] {+1} (4)
         edge node {-1} (8)
    (2) edge node [right] {+1} (1)
         edge node {+1} (4)   
    (3) edge node [right] {+1} (2)
    (4) edge node [left] {+1} (3)
    (8) edge node [left] {+1} (7)
    (5) edge node [right] {+1} (8) 
    (6) edge node [right] {+1} (5)
        edge node {-1} (3)
    (7) edge node [left] {+1} (6)
            edge node {+1} (5)  ;
\end{tikzpicture}
If we start with all the node states set to one we create a frustrated situation, otherwise the network choose one
of the two possible stable states. In such a case the presence of an environmental noise could induce the transition
to one state to another (to be studied). An external forcing breaks the symmetry.\par\noindent
We expect a transition phase as a function of the temperature: increasing the temperature the node states tends 
to be independent, but under a threshold the system should choose a stationary state. 
\par\noindent
The system can be generalize to consider the interactions of different cooperative networks (possibly with different internal
structure) that are connected by inhibitory links (in the case of connection with excitatory links we join the subnetwork in a 
single one). We can introduce a metadynamics where $\nu_k(t)$ is the state of the $k$ subnetwork and
we have a relation
\begin{equation}
\nu_k(t+\Delta t)-\nu_k(t)=\phi(\nu_k(t))-\gamma\left (H_{kj}\nu_j(t)\right )
\label{metastoc}
\end{equation}
where $H_{hk}\ge 0$ is an inhibitory connectivity matrix. $\phi(\nu_k(t))$ describes the tendency of the sub-network to increase
its activity and $\gamma$ the average decreasing of the activity due to the presence of other sub-networks.
This is an effective equation: $\nu_k$ should describe the network activity (i.e. it could be the time-average activity of the nodes
assuming that the network could be considered in a stationary state). Indeed the evolution time scale $\Delta t$ could be assumed
$\Delta t\gg 1$ so that the subnetwork states are relaxed to a stationary states. 
The structure of attraction basins of the stable states could be related to a potential in the state space if
$$
\nu_k(t+1)-\nu_k(t)=-\frac{\partial }{\partial \nu_k}\left [ \frac{\gamma}{2} \sum_{ij}\nu_i H_{ij}\nu_j+\sum_j V(\nu_j)\right ]
$$
where 
$$
\phi(\nu)=-\frac{\partial V}{\partial \nu}
$$
Since $\phi(\nu)\ge 0$ $V(\nu)$ is increasing. Then we introduce the energy
$$
E=\frac{\gamma}{2} \sum_{ij}\nu_i H_{ij}\nu_j+\sum_j V(\nu_j)
$$
and the equilibrium are the critical points of the energy. Moreover 
\begin{eqnarray}
E(t+1)-E(t)&\simeq& (\nu_k(t+1)-\nu_k(t))\frac{\partial}{\partial \nu_k}\left [\frac{\gamma}{2} \sum_{ij}\nu_i(t) H_{ij}\nu_j(t)+\sum_j V(\nu_j(t))\right ]\nonumber \\
&=& -\frac{1}{2}\frac{\partial}{\partial \nu_k}\left [ \frac{\gamma}{2} \sum_{ij}\nu_i(t) H_{ij}\nu_j(t)+\sum_j V(\nu_j(t)) \right ]^2\nonumber
\end{eqnarray}
Therefore the energy is a Ljapounov function and the system equilibria are defined by the critical points of the Energy function
corresponding to local minima and maxima.\par\noindent
Remark: the existence of the Energy implies that $H_{ij}$ is symmetric negative defined
$$
\frac{\partial^2 E}{\partial \nu_j \partial \nu_i}=\frac{\partial^2 E}{\partial \nu_i \partial \nu_j}
$$
\par\noindent
The stochastic effect has to be introduce but it is possible a thermodynamics approach and a thermodynamics equilibrium exists
according to the Maxwell-Boltzmann distribution and the detailed balance condition. This means that the whole network does not
satisfy this condition, but the metadynamic network realized a reversible Markov process.




\chapter*{Abstract}

Real living cell is a complex system governed by many process which are not yet understood: the process of cell differentiation is one of these. 
Cell  differentiation is the process in which cells of a specific type reproduces themselves and give arise to different type of cells.
Cell differentiation is governed by the so called Gene Regulatory Networks (GRNs).
A GRN is a collection of molecular regulators that interact with each other and with other substances in the cell to govern the gene expression levels of mRNA and proteins. 
Kauffman proposed for the first time in 1969 to model GRN through the so called Random Boolean Networks (RBN).
RBNs are networks in which each node can have only two possible values: 0 or 1, where each node represent a gene in GRN which can be "on" or "off".
These networks can model GRNs because the activity of one node represents the expression level of one gene among the whole regulation.

In this thesis work we make use of a mathematical model to develop and reproduce a possible Gene Regulatory Network for the process of cell differentiation.


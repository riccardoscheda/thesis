\documentclass[11pt]{article}
\usepackage[utf8]{inputenc}
\usepackage[english]{babel}
\usepackage{pdfpages}
\usepackage{hyperref}
\usepackage{float}
\usepackage{verbatim}
\usepackage{graphicx}
\usepackage{systeme}
\title{Network}
\author{Riccardo Scheda}

\begin{document}

\section{Boolean Dynamica with random couplings}

In some sense, the type of Boolean networks introduced by Kauffman can
be considered as a prototype of generic dynamical system, as they present
chaotic as well as regular behavior and many other typical structures of
dynamical systems. In the thermodynamic limit $N\to \infty$, there can be
“phase transitions” characterized by a critical line dividing chaotic from
regular regions of state space.


As we shall describe in more detail below, these models are often studied
in a version in which the couplings among the Boolean variables are picked
randomly from some sort of ensemble. In fact, they are often called N-K
models because each of the N elements composing the system, interact with
exactly K others (randomly chosen)

In addition, their coupling functions
are usually picked at random from the space of all possible functions of K
Boolean variables.
Another simplification is the binary nature of the variables under study.
Nevertheless, many systems have important changes in behavior when
“threshold” values of the dynamical variables are reached (e.g. the synapses
firing potential of a neuron, or the activation potential of a given chemical
reaction in a metabolic network).


\section{Structure of models}

Any model of a boolean networks starts from $N$ elements $\sigma_i \in \{0,1\}$. In the time stepping each node is given by a function of other $K$ elements:
\begin{equation}
\sigma_i(t+1) = \Lambda_i(\sigma_{i_1}(t),\sigma_{i_2}(t),...\sigma_{i_K}(t))
\label{eq:1}
\end{equation}

To establish completely the model it is necessary to specify:
\begin{enumerate}
\item the connectivity $K$ of each element;
\item the linkage of each element;
\item the evolution rule $\Lambda_i$ of each element.

Once these quantities have been specified, equation ~\eqref{eq:1} fully determins the dynamics of the system. In the most general cases the connectivity $K_i$ may vary for each node $\sigma_i$, but in this work we consider $K_i = K$ for all the nodes. 
Of fundamental importance is the way linkages are assigned to the elements. 

Now we can denote the whole condifuguration of the $\sigma_i$ as the state of the system $\Sigma_t$ at time $t$:
$$
\Sigma_t = \{\sigma_1(t),\sigma_2(t),...,\sigma_N(t)\}
$$
We can think of $\Sigma_t$ as an integer number which is the base-10 representation of the binary chain. Since each $\sigma_i$ has only two possible values, the number of all the possible configurations is $\Omega = 2^N$, so that $\Sigma_i$ can be thought as an integer satisyng $0\le \Sigma_t < 2^N$. This collection of integers is called \emph{the state space} of the dynamics of the network.

















\end{document}

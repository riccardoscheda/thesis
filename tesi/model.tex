\chapter{Cancer attractors}
\lhead[\fancyplain{}{\bfseries\thepage}]{\fancyplain{}{\bfseries\rightmark}}

%\newcommand{\folder}{/path/to/folder}
\newenvironment{sistema}%
{\left\lbrace\begin{array}{@{}l@{}}}%
{\end{array}\right.}
In questo capitolo viene fatta una piccola introduzione al modello classico preda-predatore di Lotka-Volterra.
\section{The model}
We consider a physical system that can be described by an weighted interaction network among nodes that can assume different
dynamical states (in the case of a gene network the states $\sigma\in [0,1]$ and we have models similar to spin models).
In the simplest case, we introduce a stochastic dynamics using the probability $p_i(t)$ that the node $i$ is in the state $\sigma_i=1$
(then $1-p_i(t)$ is the probability to get $\sigma_i=0$) and we define a linear equation for the probability evolution
\begin{equation}
\dot p_i(t)=\sum_j \mathcal{P}_{ij}p_j(t)-\gamma_i p_i(t)
\label{average}
\end{equation}
where $\mathcal{P}_{ij}$ are transition probability rates and $\gamma_i^{-1}$ defines the mean lifetime of the excited state.
The meaning of the rates $\mathcal{P}_{ij}$ is the rate at which the excited state of the node $j$ increases (or decreases if
$\mathcal{P}_{ij}<0$) the probability of a transition to the excited state of the node $i$. Since $0\le p_i\le 1$ for all $i$, this space should be invariant for the dynamics. This condition depends on the spectral properties of the matrix
\begin{equation}
\mathcal{P}_{ij}-\gamma_j\delta_{ij}
\label{matrix}
\end{equation}
associated to the system. Let consider the case $\mathcal{L}_{ij}\ge 0$ (i.e. we have no inhibitory link),
the first quadrant is clearly invariant and if we define
$$
\sum_i  \mathcal{P}_{ij}=\hat \gamma_j>0
$$
the matrix 
$$
\mathcal{L}_{ij}=\mathcal{P}_{ij}-\hat \gamma_j \delta_{ij}
$$
is a Laplacian matrix and the system (\ref{average}) can be written in the form
$$
\dot p_i(t)=\sum_j \mathcal{L}_{ij}p_j(t)-\Delta \gamma_i p_i(t)\qquad \Delta \gamma_i=\gamma_i-\hat \gamma_i
$$
and by assumption we have $\gamma_i>\hat \gamma_i$. The eigenvalues of the matrix $\mathcal{L}_{ij}$
have all negative real part except the null eigenvalue. It follows that all the eigenvalue of the matrix (\ref{matrix})
has negative real part and the dynamics is a contraction towards the origin:
a stable solution (i.e. without any external stimulus the system relaxes to the $\sigma_i=0$ state).
A non trivial stationary can be achieved only if an external stimulus is inserted
\begin{equation}
\dot p_i(t)=\sum_j \mathcal{P}_{ij}p_j(t)-\gamma_i p_i(t)+\epsilon f_i(t)
\label{average_ext}
\end{equation}
The stationary solution has to satisfy $p_i\in [0,1]$ so that $f_i(t)\ge 0$ otherwise we can have negative probability 
when $p_i\simeq 0$. The case of a Laplacian matrix
$$
\hat \gamma_i=\gamma_i
$$
we get another possible stationary solution for $\mathcal{L}_{ij}p^\ast_j=0$ in the first quadrant and
the subspace $\sum p_i=0$ is invariant and the dynamics is a contraction in this subspace (in general).
Then the system a stable stationary solution even in absence of an external stimulus.\par\noindent
The presence of inhibitory links complicates the model and one has to prove that
\begin{itemize}
\item 1) there exists a physical space: an invariant cone in the first quadrant where the dynamics is a contraction towards
the origin;\par
\item 2) the external stimulus maintains the solution in the physical space.
\end{itemize}
Another solution could be to introduce boundary conditions so that $p_i\ge 0$ in any case (the system is non linear in such a case).
\par\noindent
The eigenvalues of the matrix (\ref{matrix}) define the different relaxation time scale the process and determine its rectivity
to the change of the external stimulus: in a typical problem one consider a slowly varying external stimulus so that the
system could be considered i a quasi stationary state
$$
\sum_j \mathcal{L}_{ij}p_j-\Delta \gamma_i p_i=-\epsilon f_i(t) \qquad \frac{df_i}{dt}\ll 1
$$
the derivative is small with respect to the eigenvalues of th matrix (adiabatic approximation). On the other hand we have the effect
of a correlated noise (we need to introduce a correlation in order have a continuous function $f_i(t)$). The problem is to
study the relation between the solution and the spectral properties of the matrix $\mathcal{L}_{ij}$: we simplify the
equation by assuming $\Delta \gamma_i=\Delta \gamma$ so that if $\lambda$ is an eigenvalue of $\mathcal{L}_{ij}$
then $\lambda-\Delta \gamma$ is an eigenvalue of the matrix (\ref{matrix}) and we assume that the dynamics
is perturbed by
\begin{equation}
\dot p_i(t)=\sum_j\left ( \mathcal{L}_{ij}+\Delta \mathcal{L}_{ij}\right ) p_j(t)-\Delta \gamma p_i(t)+\epsilon f_i(t)
\label{average_p}
\end{equation}
where the perturbation $\Delta\mathcal{L}_{ij}$ is a Laplacian matrix ($\sum_i \Delta\mathcal{L}_{ij}=0$ and we
assume $<\mathcal{L}>=0$) that can
represent an error in the measure of the transition rates $\mathcal{L}_{ij}$ or possible evolution of network due to
in time. In the first case we have an ensemble of transition matrices and we have to study the eigenvalue distribution
due to perturbation and the possible presence of bifurcation phenomena. In the second case we have a stochastic 
differential equation (since $\Delta \mathcal{L}_{ij}(t)$ can be represented as a realization of a stochastic process).
The possible approach are Perturbation Theory, Random Matrix Theory and Statistical Physics Methods for random matrices.
The external signal form the environment  (the environmental node) can be considered in the adiabatic approximation
(to be justified form a biological point of view).\par\noindent
The underlying stochastic process on the graph is defined by assigning the state $\xi_i(t)\in [0,1]$ at each node $i$ according 
to a probability distribution $\pi_i(t)$ that evolves as
$$
\dot \pi_i(t)=\sum_j \mathcal{L}_{ij}\xi_j(t)-\Delta \gamma \xi_i(t)+\epsilon f_i(t)
$$
By discretizing the dynamics for a time step $\Delta t$ we have the evolution
$$
\pi_i(t+\Delta t)=\pi(t)+\sum_j \mathcal{L}_{ij}\xi_j(t)\Delta t-\Delta \gamma \Delta t\xi_i(t)+\epsilon f_i(t)\Delta t
$$
and $\xi(t+\Delta t)$ realized according to the distribution $\pi_i(t+\Delta t)$ (stochastic cellular automata).
The average dynamics is computed by
\begin{eqnarray}
\dot <\pi_i(t)>&=&\sum_j \mathcal{L}_{ij}<\xi_j(t)>-\Delta \gamma <\xi_i(t)>+\epsilon f_i(t)\nonumber \\
&=&\sum_j \mathcal{L}_{ij}p_j(t)-\Delta \gamma p_i(t)+\epsilon f_i(t)=\dot p_i(t)\nonumber
\end{eqnarray}
and we recover the average equation (\ref{average}). But the stochastic dynamics gives information on the applicability
of the average approximation and the variability at the critical states (at bifurcation of the spectrum of $\mathcal{L}$).
The stochastic dynamics can be studied for stochastic connection matrices $\mathcal{L}+\Delta \mathcal{L}$.


